% !TEX root = ../../main.tex
\documentclass[../main.tex]{subfiles}
%
\begin{document}

\begin{definition}
    Let $R_k$ be the POP of size $k$ where $1>k$.
    
    \begin{figure}[!htbp]
        \centering
        \tikzfig{../figures}{Rk}
        \caption{The POP $R_k$}
        \label{fig:Rk}
    \end{figure}
\end{definition}

It is easy to see that the avoidance set of $R_2$ contains only the identity permutations.
Observe that $R_3$ is equivalent to the POP $P_3$ introduced in Chapter \ref{chap:levels},
where we showed that its avoidance set is enumerated by the well-known Fibonacci numbers.
Gao and Kitaev \cite{gao-kitaev-2019} enumerated the avoidance sets of $R_4$ and $R_5$ in Theorems 14 and 33 of their paper respectively.
Moreover, they showed that for all $k\geq 3$, the avoidance set of $R_k$ is in one-to-one correspondence with $n$-permutations 
such that for each cycle $c$, 
the smallest integer interval containing all elements of $c$ has at most $k-1$ elements.\\

They also observed that $Av_n(R_4)$ and the set of $n$-permutations for which the partial sums of signed displacements do not exceed 2 (to be defined later)
are the same size for all $n\geq 1$, and asked if there is an interesting bijection on one set to the other.
We construct such a bijection in this chapter by analyzing the two sets in detail.

\section{Analysing \texorpdfstring{$Av_n(R_4)$}{Av-n(R-4)}}

\begin{theorem}\label{thm:av10-an}
    If $\pi$ is an $n$-permutation that avoids $R_4$ 
    and $n\geq 5$,
    then $\pi$ is sum decomposable.
    Moreover, if $12[\alpha,\beta]$ avoids $R_4$ for permutations $\alpha$ and $\beta$ 
    where $\alpha$ is sum indecomposable,
    then $\alpha$ is 1, 21, 231, 321, 312 or 2413.
    % then \[\alpha= 1,\,21,\,231,\,321,\,312\,\text{ or }\, 2413.\]
\end{theorem}
\begin{proof}
    We claim that the only simple permutations avoiding $R_4$ are 1, 12, 21 and 2413.
    Recall that any simple $n$-permutation contains the patterns 132, 213 and 312 if $n\geq 4$.
    Thus, we can write such a simple permutation as the string of factors 
    \[\alpha^{(1)} \, p \,\alpha^{(2)} \, q \,\alpha^{(3)} \, r\,\alpha^{(4)}\] 
    where $\alpha_i$ are possibly empty factors (of length 0)
    and $p$, $q$ and $r$ are factors of length 1
    where $\text{red}(p\, q\, r)=312$.
% 
    It is clear that if $\alpha^{(2)}$ and $\alpha^{(3)}$ were not empty,
    if $\alpha^{(1)}>p$ or if $\alpha^{(4)}<p$ then the permutation would contain $R_4$.
    So $\alpha^{(2)}$ and $\alpha^{(3)}$ are empty and 
    if $\alpha^{(1)}$ and $\alpha^{(4)}$ are not empty then we must have $\alpha^{(1)}<p<\alpha^{(4)}$.
    However, this implies that if $\alpha_4$ is not empty then the permutation must be sum decomposable, so $\alpha^{(4)}$ is empty.
    If $\abs{\alpha^{(1)}}=t\geq 2$, then we must have $\alpha^{(2)}_{[1,t-1]}<q$ otherwise the permutation contains $R_4$.
    This forces $\alpha^{(1)}_{[1,t-1]}$ to be the interval $[t-1]$ which would mean that the permutation is sum decomposable.
    So $\abs{\alpha^{(1)}}=1$ if it is not empty. 
    In this case we must have $q<\alpha^{(1)}<r$, which yields the permutation 2413.\\

    Finally we show that all $n$-permutations avoiding $R_4$ are sum decomposable for $n\geq 5$. 
    If $21[\beta_1,\beta_2]$ avoids $R_4$, 
    then $\beta_1$ and $\beta_2$ must have lengths 1 or 2.
    If $2413[\beta_1,\beta_2, \beta_3, \beta_4]$ avoids $R_4$ , 
    then each $\beta_i$ must be of length exactly 1
    for all $i\in[4]$.
    % 
    The only sum indecomposable permutations avoiding $R_4$ are 
    1, 21, 321, 312, 231 and 2413,
    so $\alpha$ in the theorem must be one of these.

    % we have the following possible cases:
    % \begin{enumerate}[(i)]
    %     \item $\pi = 1    \oplus \sigma$ for some $\sigma\in Av_{n-1}(R_4)$,
    %     \item $\pi = 21   \oplus \sigma$ for some $\sigma\in Av_{n-2}(R_4)$,
    %     \item $\pi = 321  \oplus \sigma$ for some $\sigma\in Av_{n-3}(R_4)$,
    %     \item $\pi = 312  \oplus \sigma$ for some $\sigma\in Av_{n-3}(R_4)$,
    %     \item $\pi = 231  \oplus \sigma$ for some $\sigma\in Av_{n-3}(R_4)$,
    %     \item $\pi = 2413 \oplus \sigma$ for some $\sigma\in Av_{n-4}(R_4)$.
    % \end{enumerate}
\end{proof}

\begin{corollary}\label{cor:av10-an}
    Let $a_n:=\abs{Av_n(R_4)}$. Then 
    \[a_n=\begin{cases}
        n!&\text{if }\,1\leq n\leq 3,\\
        12&\text{if }\, n=4,\\
        a_{n-1}+a_{n-2}+3a_{n-3}+a_{n-4}&\text{if }\, n\geq 5.
    \end{cases}\]
\end{corollary}
\begin{proof}
    The formula is clear for $n\leq 3$. 
    Since $R_4$ represents a set of $12$ permutations, $Av_4(R_k)=4!-12=12$.
    % 
    For $n\geq 5$, recall 
    that if $\pi=12[\alpha,\beta]$ for some permutations $\alpha$ and $\beta$
    where $\alpha$ is sum indecomposable,
    then $\alpha$ and $\beta$ are unique
    by Theorem \ref{thm:12,21}.
    The recursive formula then follows directly from Theorem \ref{thm:av10-an}.
\end{proof}

\section{Partial sums of signed displacements}

\begin{definition}
    The \textit{$j$th signed displacement} of an $n$-permutation $\pi$ for $j\in [n]$ 
    is defined as $\pi_j-j$.
% \end{definition}

% \begin{definition}
    The \textit{$i$th partial sum of signed displacements} of an $n$-permutation $\pi$ for $i\in [n]$ 
    is defined as $\sum_{j=1}^i \pi_j-j$, and denoted $\pi_{\Sigma}^i$. 
\end{definition}

We denote the set of $n$-permutations for which the partial sums of signed displacements do not exceed 2 by $\mathfrak{S}_n$.
For examples, refer to Figures \ref{conforming} and \ref{non-conforming}.

\begin{figure}[!htbp]
    \centering 
    \begin{tabular}{c|cccc}
        $\pi_j$     & 3 & 1 & 4 & 2\\
        $j$         & 1 & 2 & 3 & 4\\
        $\pi_j-j$   & 2 & -1& 1 & -2\\\hline 
        partial sum & 2 & 1 & 2 & 0
    \end{tabular}
    \caption{A permutation in $\mathfrak{S}_4$}
    \label{conforming}
    \bigskip
    \begin{tabular}{c|cccc}
        $\pi_j$     & 4 & 1 & 3 & 2\\
        $j$         & 1 & 2 & 3 & 4\\
        $\pi_j-j$   & 3 & -1& 0 & -2\\\hline 
        partial sum & 3 & 2 & 2 & 0
    \end{tabular}
    \caption{A permutation that is not in $\mathfrak{S}_4$}
    \label{non-conforming}
\end{figure}

\begin{lemma}\label{lemma:av10-0}
    For an $n$-permutation $\pi$ and $k\in[n]$,
    we have $\pi_{\Sigma}^k=0$ if and only if 
    $\pi_{[1,k]}$ is itself a $k$-permutation (without reducing it).
\end{lemma}
\begin{proof}
    % It is obvious that the sum of numbers in $[k]$ under any permutation is invariant,
    % so $\sigma_\Sigma^k$ is 0 for any $k$-permutation $\sigma$.\\

    % Now suppose that $\pi_{\Sigma}^k=0$ for some $k\in [n]$.
    % That is, $\sum_{i=1}^k (\pi_k -k)=0$
    % \todo{other direction}
    We know that $\pi_j$ are positive and all distinct for all $j\in [k]$. 
    So let $i_1,i_2,\dots, i_k$ be such that 
    $1\leq \pi_{i_1}<\pi_{i_2}<\dots< \pi_{i_k}\leq n$.
    and let $r_j=\pi_{i_j}-j$ for all $j\in [k]$.
    Then $r_j$ is non-negative for all $j\in [k]$, and
    \begin{align*}
        \pi_{\Sigma}^k = 0 
        \iff \sum_{j=1}^k (\pi_j -j)=0 
        &\iff \sum_{j=1}^k \pi_{i} = \sum_{j=1}^k j\\
        &\iff \sum_{j=1}^k \pi_{i_j} = \sum_{j=1}^k j\\
        &\iff \sum_{j=1}^k (j+r_j) = \sum_{j=1}^k j\\
        &\iff \sum_{i=1}^k r_i =0.
    \end{align*}
    
    \noindent The last statement is true if and only if 
    $r_j=0$ for all $j\in [k]$.
    That is, we must have $\{\pi_i\mid i\in [k]\}=[k]$,
    i.e. $\pi_{[1,k]}$ is a $k$-permutation.\\
\end{proof}

\noindent The proof of the following theorem was inspired by the proof on OEIS page on \href{http://oeis.org/A214663}{A214663}.

\begin{theorem}\label{thm:av10-bn}
    If $\pi\in \mathfrak{S}_n$ and $n\geq 5$, 
    then $\pi$ must be of the form $12[\alpha,\beta]$, 
    where \[\beta\in \{1, 21, 231, 312, 321, 3142\}\]
    and $\alpha$ is a conforming permutation $\alpha$ of length $n-\abs{\beta}$.
\end{theorem}
\begin{proof}
    For $n\geq 5$,
    observe that the number $n$ can only occur as one of the last 3 terms of a conforming $n$-permutation.
    Note also that the sum of two conforming permutations is conforming.
% 
    Let $\pi$ be a conforming $n$-permutation.
    Then we have the following cases:
    \begin{itemize}
        \item Case 1: $\pi_n=n$.
        So $\pi_{n}-n=0$,
        and $\pi$ is conforming if and only if $\pi_{\Sigma}^{k}$
        % $k$th partial sum of signed displacements 
        does not exceed 2 for all $k\in [n-1]$.
        That is, $\pi$ is of the form $\alpha\oplus 1$ where $\alpha$
        is a conforming $(n-1)$-permutation.
        
        \item Case 2: $\pi_{n-1}=n$.
        So $\pi_{n-1}-(n-1)=1$,
        and $\pi$ is conforming only if $\pi_{\Sigma}^{n-2}=1$ or 0. 
        We can further break down into cases:

        \begin{enumerate}[(a)]
            \item If $\pi_{\Sigma}^{n-2}=0$, 
            then $\pi_{[1,n-2]}$ must itself be an $(n-2)$-permutation 
            by Lemma \ref{lemma:av10-0}.
            Then we must have $\pi_n=n-1$ - 
            that is, $\pi$ is of the form $\alpha\oplus 21$ where $\alpha$
            is a conforming $(n-2)$-permutation.

            \item If $\pi_{\Sigma}^{n-2}=1$ then $\pi_{\Sigma}^{n-1}=2$,
            so $\pi_n=n-2$.
            \begin{itemize}
                \item If $\pi_{\Sigma}^{n-3}=0$,
                then $\pi_{[1,n-3]}$ must itself be an $(n-3)$-permutation
                by Lemma \ref*{lemma:av10-0}.
                Since $\pi_{n-2}=n-1$ and $\pi_n=n-2$,
                so $\pi$ must be of the form $\alpha\oplus 231$
                where $\alpha$ is a conforming $(n-3)$-permutation.
                
                \item If $\pi_{\Sigma}^{n-3}=1=\pi_{\Sigma}^{n-2}$,
                then $\pi_{n-2}=n-2$,
                which is impossible since $\pi_n=n-2$. 
                
                \item If $\pi_{\Sigma}^{n-3}=2$.
                then $\pi_{n-2}=n-3$.
                Since the partial sums of signed displacements of $\pi$ cannot exceed 2
                and $\pi_{[n-2,n]}=(n-3)\, n\,(n-2)$,
                we must have $\pi_{n-3}=n-1$.
                This means that $\pi_{\Sigma}^{n-4}=0$. 
                Then $\pi_{[1,n-4]}$ must itself be an $(n-4)$-permutation
                by Lemma \ref*{lemma:av10-0},
                and $\pi$ is of the form $\alpha\oplus 3142$
                where $\alpha$ is a conforming $(n-4)$-permutation.
            \end{itemize}
        \end{enumerate}

        \item Case 3: $\pi_{n-2}=n$.
        So $\pi_{n-2}-(n-2)=2$,
        and $\pi$ is conforming if and only if the $(n-2)$th partial sum of signed displacements is 0.
        By Lemma \ref*{lemma:av10-0},
        the factor $\pi_{[1,n-3]}$ must be an $(n-3)$-permutation itself.
        It is easy to check that $321$ and $312$ are both conforming,
        so $\pi$ is of the form $\alpha\oplus 321$ or $\alpha\oplus 312$,
        where $\alpha$ is a conforming $(n-3)$-permutation.
    \end{itemize}
    % 1) (n-1)-perm | n; 
    % 2) (n-2)-perm | n | n-1; 
    % 3) (n-3)-perm | n | n-1 | n-2; 
    % 4) (n-3)-perm | n | n-2 | n-1; 
    % 5) (n-3)-perm | n-1 | n | n-2; and 
    % 6) (n-4)-perm | n-1 | n-3 | n |n-2; 
    % other cases are excluded by the rules.

\end{proof}

\begin{corollary}\label{cor:av10-bn}
    Let $b_n:=\abs{\mathfrak{S}_n}$. Then
    \[b_n=\begin{cases}
        n!&\text{if }1\leq n\leq 3,\\
        12&\text{if }n=4,\\
        b_{n-1}+b_{n-2}+3b_{n-3}+b_{n-4}&\text{if }n\geq 5.
    \end{cases}\]
\end{corollary}
\begin{proof}
    It is easy to check that all $n$-permutations are conforming for $1\leq n\leq 3$.
    For $n=4$, there are exactly twelve conforming $n$-permutations,
    namely 1234, 1243, 1324, 1342, 1423, 1432, 2134, 2143, 2314, 3124, 3142 and 3214.
    For $n\geq 5$,
    the recursive formula for $b_n$ follows directly from Theorem \ref{thm:av10-bn}.
\end{proof}

\section{The bijection}

\begin{theorem}
    Define $\theta: Av_n(R_4)\rightarrow \mathfrak{S}_n$ as 
    \[\theta(\pi) = 
    \begin{cases}
        \pi                 &\text{ if }\abs{\pi}\in[4],\\
        \beta \oplus 3142 &\text{ if }\pi=2413\oplus \beta \text{ for some permutation }\beta,\text{ and }\\
        \beta \oplus \alpha &\text{ if } \pi=\alpha\oplus\beta \text{ for some permutations }\alpha\text{ and }\beta, \\
        &\text{ where }\alpha\neq 2413.
    \end{cases}\] 
    Then $\theta$ is a bijection.
\end{theorem}

\begin{proof}
    We know that $\theta$ is indeed a function from $Av_n(R_4)$ to $\mathfrak{S}_n$
    by Theorems \ref{thm:av10-an} and \ref{thm:av10-bn}.
    It is clear from the definition that $\theta$ is an injection.
    Corollaries \ref{cor:av10-an} and \ref{cor:av10-bn} demonstrate that 
    $\abs{\mathfrak{S}_n}=\abs{Av_n(R_4)}$ for all $n\geq 1$,
    so $\theta$ must be a bijection.  
\end{proof}
    
\end{document}
