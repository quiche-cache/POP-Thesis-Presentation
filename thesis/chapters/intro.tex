% !TEX root = /Users/keshiayap/Desktop/Math/3\ Summer\ 2020/masters/final/main/main.tex
\documentclass[../main.tex]{subfiles}
%
\begin{document}

We write $[n]$ to denote the set of integers $\{1,2,\dots,n\}$ for $n\geq 1$.
A \textit{permutation} is a bijection from $[n]$ to itself for some $n\geq 1$.
We call such a permutation an \textit{$n$-permutation}
and typically denote it by $\pi=\pi_1 \,\pi_2 \,\dots \,\pi_n$, where $\pi_i=\pi(i)$.
We say that its \textit{length} or \textit{size} (denoted $\abs{\pi}$) is $n$.
We write $S_n$ to denote the set of all $n$-permutations.
We denote the size of any set $S$ by $\# S$ or $\abs{S}$.

\section{Background}
A permutation $\pi$ \textit{contains} a \textit{pattern} $\sigma$
if and only if there is a \textit{subsequence} in $\pi$ (of the same length as $\sigma$) 
with its letters are in the same relative order as those in $\sigma$.
For instance, the pattern $312$ occurs in 42531 (as the subsequence 423), but not in 132465.
The permutations that avoid a pattern or a set of patterns make up an \textit{avoidance set}.
Avoidance sets have been studied extensively
and research in this area has important applications to numerous fields.
Examples include 
% extremal graph theory, 
sorting devices in theoretical computer science, 
Schubert varieties and Kazhdan-Lusztig polynomials,
statistical mechanics, 
the tandem duplication-random loss model in computational biology
and bijective combinatorics 
(see \cite{kitaev-textbook} and references therein).\\


A \textit{partially ordered pattern} (abbreviated \textit{POP}) is a 
\textit{partially ordered set} (\textit{poset}) that generalizes the notion of a pattern 
when we are not concerned with the relative order of some of its letters,
and therefore may represent multiple patterns.
Specifically, a POP is a poset with $n$ elements labelled $1,\,2,\,\dots,\, n$, 
for some $n\geq 1$.
For any pattern that the POP represents, 
the partial order of the elements stipulates the \textit{relative order} of letters in the pattern,
where the labels of the elements indicate the \textit{positional order} of these letters.
% 
For example, the POP $p=$ \tikzfig{./}{example} represents all the patterns of length 4 
whose first element is larger than the third element.
That is, $p$ represents the twelve patterns \[2314,\,2413,\,3124,\,3421,\,3214,\,3412,\,4213,\,4312,\,4123,\,4321,\,4132\,\text{and }\,4231.\]
% 
A POP may represent a single pattern.
For example, the pattern 3241 represented as a POP is the chain
of four elements labelled 1, 2, 3 and 4 with the order $4<2<1<3$.
Note that 3241 is the permutation inverse of 4213, and this is not a coincidence.\\
% 
A permutation \textit{contains} a POP if and only if it contains at least one of the patterns represented by that POP.
Otherwise, it \textit{avoids} the POP.
For example, the permutation 3472615 contains 21 occurrences of the POP $p$ (defined above) whereas 132456 avoids $p$.\\

\section{Motivation and structure}
Enumerating the permutations of different lengths in the avoidance set of a pattern or set of patterns  
and finding one-to-one correspondences to well-known combinatorial objects 
is a topic of great interest.
Several classical combinatorial objects may be related to a single avoidance set, 
and finding these connections would allow us to 
understand seemingly disparate objects under a common framework \cite{kitaev-textbook}.
With the aid of a computer software, Gao and Kitaev \cite{gao-kitaev-2019} 
conducted a systematic search of connections between sequences in The Online Encyclopedia of Integer Sequences (OEIS) \cite{oeis} 
and the enumeration of permutations avoiding POPs with 4 or 5 elements.
They observed connections to 38 sequences in OEIS and 
listed 15 combinatorial objects with which potentially interesting bijections might occur
with the avoidance sets of certain POPs
(see Table 6 and 7 of their paper).\\

The goal of this thesis was to find as many bijections between the pairs of objects in ways that were meaningful.
% answer as many of these bijective questions as possible.
With the help of an interactive software \href{http://www.cs.otago.ac.nz/PermLab/}{PermLab} \cite{permlab},
we successfully constructed nontrivial bijections for five of these pairs
and found generalizations them whenever possible.
The objects are listed in Table \ref{table:summary}
and the bijections are discussed in Chapters \ref{chap:av1}, \ref{chap:Q}, \ref{chap:levels} and \ref{chap:av10}.
One bijection (discussed in Section \ref{sect:juggling}) emerged directly from the original proof of the enumeration of ground-state juggling sequences by Chung and Graham \cite{chung-graham}. 
For each of the remaining four bijections, 
we had to first realize that both sets in the corresponding pair
could be partitioned into subsets of corresponding sizes.
This allowed us to construct similar recursive algorithms that can build the sets in parallel,
which in turn yielded (one or many) bijections that could be constructed directly and explicitly.
Thus, we ended up with 
% something more useful - 
a thorough understanding of the permutations that avoid each POP and of the corresponding combinatorial objects.\\

During our analysis, we discovered a set of patterns that are avoided by infinitely many \textit{simple permutations} (to be defined in Chapter \ref{chap:prelim}),
which are, in fact, enumerated by a translation of the well-known \textit{Fibonacci sequence}.
We constructed an algorithm that allows one to obtain this set of permutations recursively
and prove this in Chapter \ref{chap:fib}.
Chapter \ref{chap:prelim} defines all the relevant terms and concepts in detail
and Chapter \ref{chap:conclusion} summarises our research and lists possible avenues of further research.

% \begin{itemize}
%     \item see Page 135 of Kitaev's textbook (section 3.5)
%     \item Egge (2007): Restricted Symmetric Permutations found $S_n^{g}(R)$ for $n\geq 0, g\in D_8$ and $R\subseteq S_3$.
%     \item Troyka, Centrosymmetric permutations (2019) found 
%     \item \href{https://doi.org/10.1016/j.disc.2005.06.016}{Albert and Atkinson, Simple permutations and pattern restricted permutations (2005)}.
%     A permutation class with only finitely many simple permutations has a readily computable algebraic generating function. (as in Brignall (2008, Theorem 2.1) 
%     \item \href{https://arxiv.org/abs/1605.04297}{Vincent Vatter, Growth rates of permutation classes: from countable to uncountable}:\\
%     Proposition 2.3. The generating function for a sum closed class is $1/(1-g)$ where $g$ is the generating function for nonempty sum indecomposable permutations in the class.
%     \item \href{https://link-springer-com.proxy.queensu.ca/content/pdf/10.1007/s11856-014-1098-8.pdf}{Inflations of Geometric Grid Classes of Permutations by Albert, Ruskuc, Vatter (2013)}\\
%     Every permutation class with growth rate less than $\kappa$ has a rational generating function. This bound is tight as there are permutation classes with growth rate $\kappa \approx 2.20557$ which have nonrational generating functions
%     \item \href{https://search-proquest-com.proxy.queensu.ca/docview/2400300156?pq-origsite=primo}{Kitaev (2003): Generalized Patterns in Words and Permutations (PhD Thesis)}
%     \item \href{https://arxiv.org/pdf/1409.4368.pdf}{Cooper, Kirkpatrick: (2014) The Complexity of Counting Poset and Permutation Patterns}
%     \item \href{http://www1.chapman.edu/~jipsen/gap/posets.html}{Interactive Poset and Lattice Drawing Java Applet}
% \end{itemize}

\begin{table}
    \centering
    \begin{tabular}{|c|c|c|c|}
        \hline
            \thead{POP}  & \thead{OEIS sequence \\(beginning with $n=1$)} & \thead{Equinumerous structures} & \thead{Location} \\ \hline
            \makecell{\tikzfig{../../figures}{Av1}}  & \makecell{\href{https://oeis.org/A111281}{A111281}\\1, 2, 6, 16, 40, 100, 252,\\636, 1604, 4044,\\10196,25708, ...} & \makecell{permutations \\avoiding the patterns \\ 2413, 2431, 4213, 3412, \\ 3421, 4231, 4321, 4312}   & Chapter \ref{chap:av1} \\
            \makecell{\tikzfig{../../figures}{Av13}} & \makecell{\href{https://oeis.org/A084509}{A084509}\\1, 2, 6, 24, 96, 384, 1536,\\6144, 24576, 98304,\\393216, 1572864, ...} & \makecell{number of ground-state \\ 3-ball juggling sequences \\ of period $n$} & Section \ref{sect:juggling} \\
            \makecell{\tikzfig{../../figures}{Av5}}  & \makecell{\href{https://oeis.org/A025192}{A025192}\\1, 2, 6, 18, 54, 162, 486,\\1458, 4374, 13122,\\39366, 118098, ...} & \makecell{2-ary shrub forests\\of $n$ heaps avoiding\\the patterns 231, 312, 321}  & Section \ref{sect:shrub} \\
            \makecell{\tikzfig{../../figures}{Av9}}  & \makecell{\href{https://oeis.org/A045925}{A045925}\\1, 2, 6, 12, 25,48, 91,\\168, 306, 550, 979,\\1728, 3029...} & \makecell{levels in all \\compositions of $n+1$ \\with only ones and twos} & Section \ref{sect:av9-levels} \\
            \makecell{\tikzfig{../../figures}{Av10}} & \makecell{\href{https://oeis.org/A214663}{A214663} and \href{https://oeis.org/A232164}{A232164}\\1, 2, 6, 12, 25, 57, 124,\\268, 588, 1285, 2801,\\6118, 13362, ...} & \makecell{number of $n$-permutations \\ for which the partial sums \\of signed displacements \\ do not exceed 2}  & Chapter \ref{chap:av10} \\
        \hline
    \end{tabular}
    \caption{List of POPs studied}
    \label{table:summary}
\end{table}

\end{document}