% !TEX root = ../../main.tex
\documentclass[../main.tex]{subfiles}
%
\begin{document}

    For an $n$-permutation $\pi$, 
    we say that $\pi_{i_1}\, \pi_{i_2}\,\cdots\,\pi_{i_k}$ 
    is a \textit{subsequence} of $\pi$ if and only if 
    $1\leq i_1 < i_2 < \cdots i_k \leq n$ and $k\in [n]$.
    % 
    For an $n$-permutation $\pi$ and any $1\leq i,\, j\leq n$,
    the contiguous substring $\pi_i\pi_{i+1}\,\cdots\,\pi_j$ 
    is called a \textit{factor} of $\pi$.
    We denote $\pi_i\pi_{i+1}\,\cdots\,\pi_j$ simply as $\pi_{[i,j]}$.
    Note that if $i=j$, then $\pi_{[i,j]}=\pi_i$ has length 1
    and we call it a \textit{point}, \textit{term} or an \textit{element}.
    If $i>j$ then $\pi_{[i,j]}$ has length 0,
    and we say that it is \textit{empty}.  
    % 
    Let $\alpha=\pi_{[i_1,j_1]}$ and $\beta=\pi_{[i_2,j_2]}$ be non-empty factors of $\pi$.
    We write $\alpha<\beta$ if and only if 
    $\pi_{\ell_1}<\pi_{\ell_2}$ for all $\ell_1\in [i_1,j_1]$ and $\ell_2\in [i_2,j_2]$.\\

    Note that we may extend the definition of factors of permutations to factors of factors.
    If $\pi$ is a factor of size $n$ of a larger $m$-permutation $\zeta$,
    say $\pi=\zeta_{[i,j]}$ for some $1\leq i\leq j\leq m$,
    then we use $\pi_k$ to denote $\zeta_{i+k-1}$ for any $k\in [n]$.
    Then $\pi_{[k,\ell]} = \zeta_{[i+k-1,\, i_\ell-1]}$ for any $1\leq k\leq \ell\leq n$.\\

%     Sometimes, it may be useful to state a point of a factor $\pi$, 
%     where its exact position within $\pi$ is arbitrary.
%     We denote such a point by $\hat{\rho}$. 
    
    We say that a factor $\sigma=\pi_{[i,j]}$ (for some $1\leq i\leq j\leq n$) of an $n$-permutation $\pi$ 
    \textit{contains the number} $x$, denoted as $x\in \sigma$,
    if and only if $\pi_{\ell}=x$ for some $\ell\in [i,j]$.
    Otherwise, we say that $\sigma$ does not contain the number $x$
    and write $x\not\in \sigma$.\\

% \begin{definition} % [Interval]
    For an $n$-permutation $\pi$, we say that a factor $\pi_{[i,j]}$ 
    is an \textit{interval} if and only if it contains exactly the numbers in a contiguous interval of $[n]$.
    That is, if and only if 
    $\{\pi_\ell \mid \ell\in [i,j]\} = [s,t]$ for some $s,t\in [n]$.
    For example, the factor 2413 is an interval
    while the factor 241 is not an interval.
    An interval of an $n$-permutation is \textit{trivial}
    if and only if its length is 0, 1 or $n$.\\
% \end{definition}
% 

% \begin{definition} % [Reduction]
    % Let $\pi=\pi_1\pi_2\cdots \pi_n$ be a permutation,
    Let $\pi_{i_1}\pi_{i_2}\,\cdots\,\pi_{i_k}$ be a subsequence of an $n$-permutation $\pi$.
    The \textit{reduced} subsequence 
    $\text{red}(\pi_{i_1}\pi_{i_2}\,\cdots\,\pi_{i_k})$ 
    is defined as the $k$-permutation that is order-isomorphic to the subsequence.
    That is, $\text{red}(\pi_{i_1}\, \pi_{i_2}\,\cdots\,\pi_{i_k})=\sigma$ is the $k$-permutation
    where $\sigma_{s}<\sigma_{t}$ if and only if $\pi_{i_s}<\pi_{i_t}$
    We say that $\sigma$ is the \textit{reduction} of the subsequence $\pi_{i_1}\pi_{i_2}\,\cdots\,\pi_{i_k}$.
% \end{definition}

\begin{example}
    Let $\pi=1435726$. 
    Then the following statements are true:
    \begin{enumerate}[(a)]
        \item $1576$ is a subsequence of $\pi$
        \item $\text{red}(1576)=1243$
        \item $\pi_{[3,5]}=357$ and $\pi_{[6,7]}=26$ are factors of $\pi$   
        \item $\pi_{[2,4]}=435$ is an interval of $\pi$
    \end{enumerate}
\end{example}


\section{Simple permutations}\label{sect:intro-simple}

\begin{definition}
    An $n$-permutation is \textit{simple} if and only if 
    % it has no factors of length $2\leq k\leq n-1$ that are 
    all its intervals are trivial. That is, if and only if 
    its intervals are all of length 0, 1 or $n$.
\end{definition}

\begin{definition}
    Let $\sigma$ be a $k$-permutation,
        and for $\ell\in [k]$,
        let $\alpha^{(\ell)}$ be a permutation of length $i_\ell$.
        % $\alpha^{(1)},\, \alpha^{(2)},\,\dots,\, \alpha^{(k)}$ and $\sigma$ be permutations
        % of lengths $i_1,\, i_2,\,\,\dots,\, i_k$ and $k$ respectively.
    We define the \textit{inflation} of $\sigma$ by $\alpha^{(1)}, \alpha^{(2)},\dots, \alpha^{(k)}$ 
    as the permutation \[\pi=\sigma[\alpha^{(1)}, \alpha^{(2)},\dots, \alpha^{(k)}]\]
    of length $n := i_1+i_2+\cdots i_k$
    where given $s_0:=0$, $\ell\in [k]$, $s_\ell := i_1+i_2+\cdots i_\ell$,
    the following hold:
    \begin{itemize}
        \item the factors $\pi_{[s_{\ell-1}+1,\, s_{\ell}]}$ are intervals
        \item $\text{red}(\pi_{[s_{\ell-1}+1,\, s_{\ell}]}) = \alpha^{(\ell)}$ 
        \item $\pi_{[s_{t-1}+1,\, s_t]}<\pi_{[s_{u-1}+1,\, s_u]}$ if and only if $\sigma_t < \sigma_u$.
    \end{itemize}
    % 
    We call $\sigma$ a \textit{quotient} of $\pi$.
    %  and $\pi$ the \textit{deflation} of $\sigma[\alpha^{(1)}, \alpha^{(2)},\dots, \alpha^{(k)}]$.
\end{definition}

\begin{example}
    The permutation 526314 is simple while the 4215763 is not.
    The following statements are true: 
    \begin{enumerate}[(a)]
        \item $4215763=3142[1,21,132,1]$. Note 4, 21, 576 and 3 are intervals of 4215763.
        \item 3142 is a quotient of 4215763
        \item 4215763 is an inflation of 3142 
    \end{enumerate}
\end{example}

% Prop 8.1.6 in the book by Kitaev
\begin{prop}[Albert and Atkinson (2010) \cite{albert-atkinson}]
    Every permutation may be written as the inflation of a unique simple permutation. 
    Moreover, if $\pi$ can be written as $\sigma[\alpha^{(1)},\, \alpha^{(2)},\, \dots,\, \alpha^{(m)}]$ 
    where $\sigma$ is simple and $m\geq 4$, 
    then the $\alpha^{(i)}$s are unique.
\end{prop}

\section{Separable permutations}

\begin{definition}
    Suppose $\pi$ and $\sigma$ are permutations of length $n$ and $m$ respectively.
    We define the \textit{direct sum} (or simply, \textit{sum}), using the operator $\oplus$, and the skew sum, using the operator $\ominus$,
    of $\pi$ and $\sigma$ as the permutations of length $m+n$ as follows:
    \[\pi \oplus \sigma = 12[\pi,\sigma]
    \quad \text{and}\quad
    \pi \ominus \sigma = 21[\pi,\sigma].\]
    % \begin{align*}
    %     (\pi \oplus \sigma)_i & = \begin{cases}
    %         \pi_i &\text{if }1\leq i\leq m,\\
    %         \sigma_{i-m}+m  &\text{if }m+1\leq i\leq m+n,
    %     \end{cases}\\
    %     (\pi \ominus \sigma)_i & = \begin{cases}
    %         \pi_i +n &\text{if }1\leq i\leq m,\\
    %         \sigma_{i-m}  &\text{if }m+1\leq i\leq m+n,
    %     \end{cases}
    % \end{align*}
    \end{definition}

\begin{definition}
    If a permutation is an inflation of 12 or 21, 
    we call it 
    \textit{sum decomposable} and \textit{skew sum decomposable} respectively.
    If a permutation is not sum decomposable we say it is \textit{sum indecomposable},
    and if it is not skew sum decomposable we say it is \textit{skew-sum indecomposable}.
\end{definition}

\begin{definition} % [Separable permutations]
    A permutation is \textit{separable} if it can be obtained
    by repeatedly applying the $\oplus$ and $\ominus$ operations on the permutation 1.
\end{definition}

\begin{example}
    The permutation 587694231 is separable, since 
    \begin{align*}
        587694231
        &=14325\ominus 4231\\
        &=(1\oplus 3214)\ominus (1\ominus 231)\\
        &=(1\oplus (321\oplus 1)\ominus (1\ominus (12\ominus 1)))\\
        &=(1\oplus ((1\ominus 21)\oplus 1)\ominus (1\ominus (12\ominus 1)))\\
        &=(1\oplus ((1\ominus (1\ominus 1))\oplus 1)\ominus (1\ominus ((1\oplus 1)\ominus 1))).
    \end{align*}
\end{example}
    
\begin{example}
    All permutations of length 3 are separable.
    The only permutations of length 4 that are are not separable are 2413 and 3142.
\end{example}

% We state the following well-known theorems without proof,
% that are very useful in enumerating and understanding the avoidance sets of certain patterns:

% (Theorem 2.2.36 of Kitaev's textbook)
\begin{theorem}[\textit{folklore}]\label{thm:separable}
    The separable permutations are those avoiding 2413 and 3412.
\end{theorem}

% Prop 8.1.6 and 8.1.7 in the book by Kitaev
\begin{prop}[Albert and Atkinson (2010) \cite{albert-atkinson}]\label{thm:12,21}
    If $\pi$ is an inflation of 12, 
    say $\pi = 12[\alpha,\, \beta]$,
    then $\alpha$ and $\beta$ are unique if $\alpha$ is sum indecomposable.
    The same holds with 12 replaced by 21 and ``sum'' replaced by ``skew sum''.
\end{prop}

\begin{corollary}\label{cor:separable}
    All simple permutations of length at least 4 must contain the patterns 132, 213, 231 and 312.
\end{corollary}
\begin{proof}
    It is clear that simple permutations are not separable,
    so by Theorem \ref*{thm:separable}, they must contain at least one of 2413 or 3412, 
    both of which contain all the patterns listed. 
\end{proof}

\section{Pattern/POP containment and avoidance}\label{sect:intro-avoidance}

\begin{definition} % [Containment and avoidance of patterns]
    A \textit{pattern} is a permutation of length at least 2.
    We say that a permutation $\pi$ \textit{contains} a pattern $p$ if and only if 
    there exists some subsequence $\pi_{i_1}\, \pi_{i_2}\,\cdots\,\pi_{i_k}$ of $\pi$ 
    where $\text{red}(\pi_{i_1}\, \pi_{i_2}\,\cdots\,\pi_{i_k})=p$. 
    That is, $p_j<p_\ell$ if and only if $\pi_{i_j}<\pi_{i_\ell}$
    for all $j,\, \ell\in [k]$. 
    Otherwise, we say that $\pi$ \textit{avoids} $p$.
    % 
    If $P$ is a set of patterns, we say that $\pi$ \textit{contains} $P$ 
    if $\pi$ contains any pattern in $P$. Otherwise we say that $\pi$ \textit{avoids} $P$.
\end{definition}

% \begin{definition}[Containment and avoidance of POPs]
%     We say that a graph $(V',E')$ is a \textit{sub-POP} of a POP $(V,E)$ if 
%     $V'\subseteq V$, $E'\subseteq E$ and $(V',E')$ satisfies the definitions of a POP.\\

%     We say that a POP $(V,E)$ \textit{contains} a \textit{sub-POP} $(V',E')$ 
%     if there exists a sub-POP $(V'', E'')$ of $(V,E)$ 
%     where $(V'', E'')$ and $(V', E')$ are isomorphic as POPs.
%     Otherwise, we say that $(V,E)$ \textit{avoids} $(V',E')$.
% \end{definition}

A partially ordered pattern generalizes the notion of a pattern whereby 
the order between certain elements do not have to be considered.
We are left with a partial order on the elements, 
which we can represent using a labelled partially ordered set.
Recall that a \textit{partial order} is a binary relation $\leq $ over a set $P$ 
that is \textit{reflexive}, \textit{antisymmetric} and \textit{transitive}. 
That is, for all $a,b,c\in P$, the following hold:
\begin{enumerate}[1.]
    \item $a\leq a$ (reflexivity);
    \item If $a\leq b$ and $b\leq a$ then $a=b$ (antisymmetry);
    \item If $a\leq b$ and $b\leq c$ then $a<c$ (transitivity).
\end{enumerate}
A set $P$ with a partial order $\leq$ is called a \textit{partially ordered set (poset)}, denoted $(P,\leq)$.
We may write $b\geq a$ as an equivalent statement to $a\leq b$ for any $a,\, b\in P$.
We write $a<b$ to mean that $a\leq b$ and that $a$ and $b$ are distinct. 

\begin{definition} % [Partially Ordered Pattern (POP)]
    A \textit{partially ordered pattern (POP)} $p$ of \textit{size} $k$ is a poset
    with $k$ elements labelled $1,\,2,\,\dots,\, k$.
    A POP can be expressed in one-line notation by indicating its size and 
    the minimal set of relations that defines the respective poset.
\end{definition}

\begin{definition}
    An $n$-permutation $\pi$ \textit{contains} such a POP $p$ if and only if 
    % there exist $k$ integers $1\leq i_1<i_2<\cdots <i_k\leq n$ 
    $\pi$ has a subsequence $\pi_{i_1}\pi_{i_2}\cdots \pi_{i_k}$
    such that $\pi_{i_j} < \pi_{i_m}$ if $j<m$ in the poset $P$. 
    Otherwise, we say that $\pi$ \textit{avoids} $p$.
\end{definition}

% A POP represents a set of patterns whereby
% the labels of points denotes the positional ordering within the patterns
% while the partial order indicate their relational ordering. 
% The following examples makes this clear:

\begin{example}
    The pattern 3241 represented as a POP is the 4-element chain with its elements labelled 1, 2, 3 and 4, where $4<2<1<3$.
    Note that the permutation 4213 is the inverse of 3241.
\end{example}

Recall that a poset can be represented visually as a \textit{Hasse diagram}.
A Hasse diagram of a finite poset is a visual representation of the elements and relations in the poset,
where only the \textit{covering relations} are shown. 
Recall that a covering relation in a poset $(P,\leq )$ is a binary relation 
$i\prec j$ for some $i$ and $j$ in $P$ 
where $i<j$ and there does not exist any $k\in P$ such that both $i<k$ and $k<j$ hold.
A Hasse diagram uniquely determines the partial order.\\

A Hasse diagram of a poset with $n$ elements can also be understood as a simple directed graph $(V,E)$ with an implicit upward orientation
where $V$ is a set of $n$ vertices and $E$ is a set of ordered pairs of distinct elements in $V$,
i.e. $E\subseteq \{(i,j)\mid i,j\in V,\, i\neq j\}$, that satisfies the following three conditions: 
\begin{enumerate}[1.]
    \item if $(i,j)$ is in $E$ then $(j,i)$ is not in $E$ (antisymmetry),
    \item if $(i,j)$ and $(j,k)$ are in $E$ then $(i,k)$ is not in $E$ (transitive reduction),
    \item if $(i,j)$ is in $E$ then $i\leq j$ is a relation in the poset. 
    % \item $(i,j)\in E\implies \not\exists k\in V$ such that $(i,k)\in E$ and $(k,j)\in E$.
\end{enumerate}
A POP is a labelled poset, and can therefore be represented visually as a labelled Hasse diagram.
That is, as a graph $(V,E)$ defined as above where the vertices in $V$ are labelled $1,\,2,\,\dots,\, k$.

\begin{example}\label{eg:POP-example}
    The POP of size 4 where $1>3$ is illustrated in Figure \ref{fig:POP-example}.
    It represents the patterns 2314, 2413, 3124, 3421, 3214, 3412, 4213, 4312, 4123, 4321, 4132, 4231.
    The permutation 3472615 contains 21 occurences of the POP whereas 132456 avoids it.
    % 3 47 2 615 -> 6
    % 3 4726 1 5 -> 4
    % 4 7 2 615 -> 3
    % 4 726 1 5 -> 3
    % 7 2 6 15 -> 2
    % 7 26 1 5 -> 2
    % 2 6 1 5 -> 1
\end{example}

\begin{figure}[!htbp]
    \centering
    \tikzfig{./}{example}
    \caption{The Hasse diagram representation of the POP in Example \ref{eg:POP-example}}
    \label{fig:POP-example}
\end{figure}

\begin{definition}
    Let $P$ be a pattern, a set of patterns, or a POP. 
    We denote $Av(P)$ as the set of permutations that avoid $P$ (called the \textit{avoidance set of $P$}),
    and $Av_n(P)$ as the set of $n$-permutations that avoid $P$.
    That is, $Av_n(P):=Av(P)\cap S_n$.
\end{definition}

\begin{definition}
    Let $P_1$ and $P_2$ each be a set of patterns of a POP. We say that $P_1$ and $P_2$ are 
    \textit{Wilf-equivalent} if and only if $\abs{Av_n(P_1)}=\abs{Av_n(P_2)}$ for all $n\geq 1$.
\end{definition}

It is not hard to check that containment is a partial order on any set of permutations.
In the literature, sets of permutations which are \textit{closed downward} under this order are called 
\textit{permutation classes}, or sometimes just \textit{classes}. 
That is, $\mathcal{C}$ is a \textit{permutation class} if and only if 
for any $\pi\in \mathcal{C}$ and any $\sigma$ contained in $\pi$,
we have $\sigma\in \mathcal{C}$.
If a permutation $\pi$ avoids a pattern $p$, 
then every reduced subsequence of $\pi$ avoids $p$.
In other words, every pattern contained in $\pi$ avoids $p$.
Therefore $Av(p)$ and $Av_n(p)$ are permutation classes.
The same is true if $p$ is a set of patterns or a POP.\\

% In other words, if $\pi$ avoids $p$ and contains $\sigma$, 
% then $\sigma$ must also avoid $p$.
% In the literature, $P$ is typically called the \textit{basis} of $Av_n(P)$ and $Av(P)$,

Observe that if $k_P$ is the length of the shortest pattern in a set of patterns $P$, 
then all permutations of length less than $k_P$ avoid $P$.
This means that $\abs{Av_n(P)}=n!$ for all $n<k_P$.
Therefore it suffices to enumerate $Av_n(P)$ for $n\geq k_P$
for every POP or set of patterns $P$ discussed in subsequent chapters.

% \begin{definition}
%     We say that a permutation $\pi$ \textit{contains} a POP $G$ 
%     if and only if the POP representation of $\pi$, $G_\pi$, contains $G$.
%     Otherwise, we say that $\pi$ \textit{avoids} $G$.\\
    
%     Likewise, we say that a permutation $\pi=\pi_1 \pi_2\dots \pi_n$ \textit{contains} a POP $G=(\{b_1,b_2,\dots,b_m\},E)$ 
%     if and only if $\pi$ contains a subsequence $\pi_{i_1} \pi_{i_2} \dots \pi_{i_m} $ where 
%     $(j,k)\in E \implies \pi_{i_j} <\pi_{i_k}$.
%     Otherwise, we say that $\pi$ \textit{avoids} $G$.\\
% \end{definition}

% Note: This definition does not allow us the flexibility to specify whether any two points must be adjacent in a pattern that contains it (vincular patterns).
% To allow for this, we can alter the definiition to $(V,E,\pi)$ where 
% \begin{itemize}
%     \item $V$ is a set of labelled points $V=\{a_1, a_2, \dots, a_n\}$
%     \item $E$ is a set of ordered pairs $E\subseteq \{(a_i,a_j)\mid a_i\neq a_j\}$ where
%     \item $(a_i,a_j), (a_j,a_k)\in E\iff (a_i,a_k)\in E$ and
%     \item $(a_i,a_j)\in E\implies (a_j,a_i)\not\in E$.
%     \item $\pi$ is a sequence of points in $V$ where subsequences may be underlined, indicating that they need to be adjacent in the permutation that includes it
% \end{itemize}

% \begin{definition}
%     The \textit{Hasse diagram representation} of a POP $\pi$ is graph $(V,E')$ with an implicit upward orientation
%     where $E'\subseteq E$ and 
%     $(i,j)\in E'\implies \not\exists k\in V$ such that both $ (i,k)\in E'$ and $(k,j)\in E'$.
% \end{definition}

% \begin{definition}[Isomorphic POPs]
%     We say that two POPs $(V=\{a_1,a_2,\dots,a_n\},E)$ and $(V'=\{b_1,b_2,\dots,b_m\},E')$ are \textit{isomorphic as POPs} if the following hold:
%     \begin{itemize}
%         \item $n=m$, 
%         \item $(a_i, a_j)\in E \iff (b_i, b_j)\in E'$.
%     \end{itemize}
%     Otherwise, they are \textit{non-isomorphic} as POPs.
% \end{definition}

% \begin{definition}[POP representation of a permutation]
%     Let $\pi=\pi_1 \pi_2\dots \pi_n$ be a permutation represented using one-line notation. 
%     The \textit{POP representation} of $\pi$ is 
%     is a POP $(V=\{b_1,b_2,\dots,b_n\}, E)$ (denoted $G_\pi$) where $b_{a_i} = i$. 
% \end{definition}

% As an example, we will prove a simple proposition that will be useful in the following chapters.
% 
% \begin{prop}
%     If $\sigma$ is a simple permutation then $\sigma$ contains 213.
% \end{prop}
% \begin{proof}
%     If $ \sigma=\sigma_1\sigma_2\cdots \sigma_n$ is simple, then it cannot be an inflation of 12 or 21.
%     Thus $\sigma_1\neq 1, n$ and $\sigma_n\neq 1,n$.
%     So there exists $i,j\neq 1,n\st\sigma_i=1, \sigma_j=n$.
%     If $i<j$ then $\text{red}(\sigma_1\sigma_i \sigma_j)=213$. \\

%     So $j<i$. If there exist $s,t\in\{1,2,\dots, j\}\st \sigma_s>\sigma_t, s<t$, then $\text{red}(\sigma_s\sigma_t\sigma_j)=213$.
%     Thus $\sigma_1\sigma_2\cdots \sigma_j$ is an ascending sequence.\\
    
%     Let $\sigma_1=k$. 
%     If $\sigma_t>k$ for any $t>i$, then $\text{red}(\sigma_1\sigma_i\sigma_t)=213$.
%     So $\sigma_t<k$ for all $t>i$.\\

%     Let $\sigma_s=k+1$. 
%     If there exists some $r<s$ such that $\sigma_r<k$, then $\text{red}(\sigma_1\sigma_r\sigma_s)=213$.
%     So $\sigma_r\geq k$ for all $r=1,2,\dots, s$, which means that $\sigma_1 \sigma_2\cdots\sigma_s$ is an interval, 
%     but this contradicts the assumption that $\sigma$ is simple.\\

%     \begin{center}
%         \begin{tabular}{|llllll|lll|}\hline
%                    &     &     & $\sigma_j=n$ &     &              &  &            &  \\
%                    &     & \reflectbox{$\ddots$} &            & $\ddots$ &              &  &            &  \\
%                    & \reflectbox{$\ddots$} &     &            &     & $\sigma_s=k+1$ &  &            &  \\
%         $\sigma_1=k$ &     &     &            &     &              &  &            &  \\\hline
%                    &     &     &            &     &              & $\cdots$ &            & $\cdots$ \\
%                    &     &     &            &     &              &  & $\sigma_i=1$ & \\\hline
%         \end{tabular}
%     \end{center}
% \end{proof}
% Notes:
% In fact, we proved that for any simple permutation, there is an occurence of 213 where the '1' entry in the permutation is in fact 1.
% This suggest that there might be 'many' occurences of 213 in any simple permutation.\\

\section{Matrix representations of permutations}
% Should have changed it all to using the following convention!

% --------------------------------------------- %
% For an $n$-permutation $\pi$,
% its \textit{permutation matrix}
% is a binary $n\times n$ matrix, denoted $M(\pi)$ where 
% \[M(\pi)_{i,j}=1 \iff \pi_i=j.\]
% Moreover, its \textit{pattern matrix}
% is an $n\times n$ matrix, denoted $M'(\pi)$, where 
% \[M'(\pi)_{i,j}=\begin{cases}
%     i&\text{ if }\pi_i=j,\\
%     0&\text{ otherwise}.
% \end{cases}\]
% We may omit displaying the 0s and brackets
% if no confusion would arise.
% When illustrating these matrices, 
% we index the bottom leftmost entry by $(1,1)$, 
% the top leftmost entry by $(1,n)$ 
% and the bottom rightmost entry by $(n,1)$.
% --------------------------------------------- %

\begin{definition}
    For an $n$-permutation $\pi$,
    its \textit{permutation matrix}
    is a binary $n\times n$ matrix, denoted $M(\pi)$ where 
    \[M(\pi)_{n-i+1,j}=1 \iff \pi_j=i.\]
    Moreover, its \textit{pattern matrix}
    is an $n\times n$ matrix, denoted $M'(\pi)$, where 
    \[M'(\pi)_{n-i+1,j}=\begin{cases}
        i&\text{ if }\pi_j=i,\\
        0&\text{ otherwise}.
    \end{cases}\]
    We may refer to the non-zero entries in a permutation matrix or pattern matrix as \textit{points}.
    Sometimes, we may omit displaying the 0s and the traditional brackets
    if no confusion would arise.

\end{definition}
\begin{example}
    Let $\pi=312$. Its permutation matrix is
    \[M(\pi)\quad
    =\quad\begin{pmatrix}
        1&0&0\\
        0&0&1\\
        0&1&0
    \end{pmatrix}\quad
    =\quad\begin{matrix}
        1& & \\
         & &1\\
         &1& 
    \end{matrix}\] 
    and its pattern matrix is
    \[M'(\pi)\quad
    =\quad\begin{pmatrix}
        3&0&0\\
        0&0&2\\
        0&1&0
    \end{pmatrix}\quad
    =\quad\begin{matrix}
        3& & \\
         & &2\\
         &1& 
    \end{matrix}.\] 
\end{example}

\begin{definition}
    The \textit{weight} of a matrix is the number non-zero entries it contains,
    and we denote the weight of a matrix $A$ by $\abs{A}$. 
\end{definition}

\begin{example}
    The weight of the permutation matrix of $\pi$ 
    is equal to the length of $\pi$.
    The weight of any column or row of a permutation matrix is 1.
\end{example}

\section{Lattice matrices}
 
\begin{definition}
    We call a matrix (or submatrix) \textit{void} if it has no rows or no columns.
    A matrix (or submatrix) is \textit{trivial} if its weight is 0, and \textit{nontrivial} otherwise.
    That is, a trivial matrix is either void or is a zero matrix.
    All void matrices are trivial.
\end{definition}

If we know that a permutation contains a certain pattern,
it might be helpful to represent its permutation as a block matrix 
in order to prove some claims about the permutation.
We first show an example before stating a formal definitions:\\

Suppose we know that the $n$-permutation $\pi$ contains the pattern $p:=312$.
That is, there exist $i$, $j$ and $k$ where $1\leq i<j<k\leq n$ and $\pi_i\, \pi_j\, \pi_k$ reduces to 312.
The columns $i$, $j$ and $k$ partition the rest of the permutation matrix $M(\pi)$ into 4 (possibly trivial) blocks of columns,
and the rows $\pi_i,\, \pi_j$ and $\pi_k$ partition the rest of the permutation matrix $M(\pi)$ into 4 (possibly trivial) blocks of rows.
This gives rise to another representation of $M(\pi)$ as a $7\times 7$ block matrix.
In this representation we can find the $3\times 3$ matrix $M(p)$
interwoven with a $4\times 4$ block matrix $(\alpha_{ij})_{i,j\in [4]}$
as depicted in Figure \ref{fig:lattice},
with the following alterations:
\begin{itemize}
    \item the ones in columns $i,$ $j$ and $k$ are replaced by $\pi_i$, $\pi_j$ and $\pi_k$ respectively, \\
    (in other words, we replace the submatrix corresponding $M(p)$ with the pattern matrix $M'(p)$)
    \item the zeroes in columns $i,$ $j$ and $k$ that are also in row $\pi_i$, $\pi_j$ or $\pi_k$ 
    are replaced by plus signs, 
    \item the trivial blocks in columns $i,$ $j$ and $k$ are replaced by vertical bars,
    \item the trivial blocks in rows $\pi_i$, $\pi_j$ and $\pi_k$ are replaced by horizontal bars, and finally,
    \item the conventional matrix brackets are omitted.
\end{itemize} 

\noindent Note that we may also alter the ones, zeroes and $\alpha_{ij}$ blocks (where $i,j\in[n+1]$)
differently based on what properties of the permutation we are trying to highlight.
This figure is reminiscent of the lattice structure of gridded window panes,
so we call it the \textit{$p$-lattice matrix of $\pi$}, or simply a \textit{lattice matrix},
and denote it as $L_p(\pi)$.\\

\begin{figure}
    \begin{center}
        \begin{tabular}{ccccccc}
            $\alpha_{11}$ &   \Big |  & $\alpha_{12}$ &   \Big |  & $\alpha_{13}$ &   \Big |  & $\alpha_{14}$\\
            --------      &     3     &    --------   &  $\Plus$  &    --------   &  $\Plus$  & --------\\
            $\alpha_{21}$ &   \Big |  & $\alpha_{22}$ &   \Big |  & $\alpha_{23}$ &   \Big |  & $\alpha_{24}$\\
            --------      &  $\Plus$  &    --------   &  $\Plus$  &    --------   &      2    & --------\\
            $\alpha_{31}$ &   \Big |  & $\alpha_{32}$ &   \Big |  & $\alpha_{33}$ &   \Big |  & $\alpha_{34}$\\
            --------      &  $\Plus$  &    --------   &      1    &    --------   &  $\Plus$  & --------\\
            $\alpha_{41}$ &   \Big |  & $\alpha_{42}$ &   \Big |  & $\alpha_{43}$ &   \Big |  & $\alpha_{44}$\\
        \end{tabular}
        \caption{The lattice matrix $L_{312}(\pi)$}
        \label{fig:lattice}
    \end{center}
\end{figure}

\noindent In general, we may use the following definition:

\begin{definition}
Consider a permutation $\pi$ on $n$ letters and choose
$m$ indices $1\leq i_1<i_2<\cdots i_m\leq n$.
Write $I=(i_1,\, i_2,\, \dots,\, i_m)$ and let $p:=\text{red}(\pi_{i_1}\pi_{i_2}\, \cdots\, \pi_{i_m})$.
We proceed to define the lattice matrix $L_p(\pi)$.
Put $i_0=0$ and $i_{m+1}=n+1$.  
We use the values $i_1,\, i_2,\, \dots,\, i_m$ to partition the column indices into subintervals
and the values $\pi_{i_1},\,\pi_{i_2},\,\dots,\,\pi_{i_m}$  
to partition the row indices into subintervals.\\

A \textit{block} in $L_p(\pi)$ is a (possibly trivial) continguous block submatrix of the permutation matrix $M(\pi)$
with its column and row indices each given by a relevant subinterval defined above. 
To make this more explicit, we note that $\pi(i_{p\inv(1)})< \pi(i_{p\inv(2)}) < \cdots < \pi(i_{p\inv(m)})$.  
Write $j_k:= \pi(i_{p\inv(k)})$.
Put $j_0= 0$ and $j_{m+1}=n+1$.
Then $M_{S\times T}$ where $S=[i_{s-1}+1,\, i_s -1]$ and $T=[j_{t-1}+1,\, j_t -1]$
is a \textit{block}
for all $s$ and $t$ in $[m+ 1]$.
We label the block $M_{S\times T}$ as $\alpha_{s,t}$.
Note that $\alpha_{s,t}$ is void if either $i_s=i_{s-1}$ or $j_t=j_{t-1}+1$.
We may write $\alpha_{s,t}$ as $\alpha_{st}$ if no confusion would arise.
Note that $M_{i_k, \pi(i_k)}$, $M_{i_k\times T}$ and $M_{S\times j_{k}}$ may also be referred to as blocks 
for any $k\in [m]$ and subintervals $S$ and $T$ defined as above.\\

When depicting $L_p(\pi)$, we replace the zero entries in 
column $i_k$ by a vertical line and the zero entries in row $j_k$ by a horizontal line for every $k\in [m+1]$. 
We also omit the traditional parentheses around the entire matrix $L_p(\pi)$.
% We may replace the 1 in row $i_k$ and column $\pi(i_k)$ by the integer $p(k)$.\\ 
\end{definition}


\begin{prop}
    It is easy to deduce the following properties of the lattice matrix $L_p(\pi)$ defined above:
    \begin{enumerate}[1.]
        \item $(\alpha_{s,t})_{a,b} = M_{j_{s-1}+a,\, i_{t-1}+b}$
        for $a \in [j_s - j_{s-1}-1]$ and $b\in [i_{t} - i_{t-1}-1]$
        \item Each horizontal (respectively, vertical) bar is either void, or is a single row (respectively, column) of 0s.
        \item All block matrices in the same row (respectively, column) of the lattice matrix have the same number of rows (respectively, columns).
        \item Any square block submatrix of the lattice matrix has the same total number of rows as columns.
    \end{enumerate}
\end{prop}

We would like to describe the relationships between points in different blocks in a lattice matrix precisely.
The following definitions provide an intuitive way to do so.

\begin{definition}
    Let $p$ and $\pi$ be permutations of length $m$ and $n$ respectively where $m\leq n$ and $\pi$ contains $p$.
    Let $L_p(\pi)$ be the $p$-lattice matrix of $\pi$ with its blocks denoted by $\alpha_{ij}$ for $i,\, j\in [m+1]$.
    
    \begin{enumerate}[(a)]
        \item For the points in $L_p(\pi)$ that correspond to the pattern $p$, we define a point being \textit{adjacent} to a block in a natural way.
        For example, in Figure \ref{fig:lattice}, the point labelled 1 is adjacent to the blocks $\alpha_{32},\,\alpha_{33},\,\alpha_{42}$ and $\alpha_{43}$
        while the point labelled 2 is adjacent to the blocks $\alpha_{23},\,\alpha_{24},\,\alpha_{33}$ and $\alpha_{34}$.
        We note that every point is adjacent to exactly four blocks.
        % \item We say that a block $\alpha_{ij}$ is \textit{adjacent} to a block $\alpha_{k\ell}$ if and only if 
        % $i\in [k-1,k+1]$ and $j\in [\ell-1,\ell+1]$ and $(i,j)\neq (k,\ell)$.
        \item We use the four cardinal directions, north, south, east and west to 
        to indicate where one point lies in relation to another in the visual representation of the permutation matrix $M(\pi)$.
        Explicitly, a point is \textit{north} (respectively, \textit{south}) of another point if and only if the row index in $M(\pi)$ of the former point is smaller than that of the latter,
        and is \textit{west} (respectively, \textit{east}) of another point if and only if the column index in $M(\pi)$ of the former point is smaller than that of the latter.
        \item For $i_1,i_2 \in [m+1]$, we say that $\alpha_{i_1j}$ is \textit{to the left} (respectively, \textit{to the right}) of $\alpha_{i_2j}$ 
        if and only if all the points in $\alpha_{i_1j}$ are west (respectively, east) of all the points in $\alpha_{i_2j}$. 
        \item For $j_1,j_2 \in [m+1]$, we say that $\alpha_{ij_1}$ is \textit{superior} (respectively, \textit{inferior} to $\alpha_{ij_2}$ 
        if and only if all nontrivial columns of $\alpha_{ij_2}$ are north (respectively, south) of all nontrivial columns of $\alpha_{ij_2}$. 
        \item We say that $\alpha_{ij}$ is \textit{leftmost} (respectively, \textit{rightmost})
        if and only if the first (respectively, last) column of $\alpha_{ij}$ is nontrivial.
        \item We say that $\alpha_{ij}$ is \textit{topmost} (respectively, \textit{bottommost})
        if and only if the first (respectively, last) row of $\alpha_{ij}$ is nontrivial.
        % \item We say that a submatrix $A$ of $L_p(\pi)$ `contains' the string $\sigma_1\sigma_2\cdots \sigma_k$
        % over the alphabet of matrices $\{\alpha_{ij}, (p_s)\mid i,j\in [m+1], s\in[m] \}$
        % if and only if $A$ contains distinct submatrices $A_1,A_2,\dots, A_k$
        % such that $A_\ell=\sigma_\ell$ for all $\ell\in [k]$.
        % \remark{do we need to specify the order of the columns of the $\sigma_i$s? if so, how?}
    \end{enumerate}
\end{definition}
\end{document}