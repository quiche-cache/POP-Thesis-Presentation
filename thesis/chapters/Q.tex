% !TEX root = ../../main.tex
\documentclass[../main.tex]{subfiles}
%
\begin{document}

\begin{definition}
    Let $Q_k$ be the POP with $k$ elements where $1>t$ for $t\in [2,n]$. 
    The Hasse diagram of $Q_k$ is the following: 
    \begin{figure}[!htbp]
        \centering
        \tikzfig{./}{Qk}
        \caption{The POP $Q_k$}
        \label{fig:Qk}
    \end{figure}
\end{definition}

Gao and Kitaev \cite{gao-kitaev-2019} enumerated the avoidance set of $Q_k$ in Theorem 2 of their paper
and observed that the sequences $\abs{Av_n(Q_4)}_{n\geq 1}$ and $\abs{Av_n(Q_5)}_{n\geq 1}$
are listed in the OEIS database as \href{https://oeis.org/A025192}{A025192} and \href{https://oeis.org/A084509}{A084509} respectively.
These sequences enumerate the 2-ary shrub forests of $n$ heaps avoiding the patterns 231, 312 and 321,
and the ground state 3-ball juggling sequence of period $n$ respectively. 
These objects will be defined in the following sections.\\
% Gao and Kitaev sought interesting bijections between the relevant sets.\\

We show that there are intimate connections between the relevant sets by constructing explicit bijections.
In fact, we construct an explicit bijection from $Av_n(Q_{k})$ to ground state $(k-2)$-ball juggling sequence of period $n$
which yields a bijection between the the original two sets as a special case.
The study that led us to this bijection also produced a new way of enumerating $Q_k$ using the concept of matrix permanents.

% by the enumeration of ground state juggling sequences.
% Finally we analyse the set $Av_n(Q_4)$ and construct an explicit bijection to 
% the 2-ary shrub forests of $n$ heaps avoiding the patterns 231, 312 and 321.

\section{Enumeration of \texorpdfstring{$Av_n(Q_k)$}{Avn(Qk)}}

We will enumerate $Av_n(Q_k)$ for $k,\, n\geq 1$ using the concept of matrix permanents.
Recall that the permanent is defined on square matrices and is similar to the definition of the determinant of a matrix,
where the signs of the summands are all positive instead of alternating.
We restate Gao and Kitaev's proof first as a reference for comparison. 

\begin{theorem}[Gao and Kitaev (2019) \cite{gao-kitaev-2019}]\label{thm:avQk}
    For $n\geq k$, $\abs{Av_n(Q_k)} = (k-1)!\times (k-1)^{n-k+1}$. 
\end{theorem}
\begin{proof}
    We proceed by induction.
    It is clear that all $(k-1)$-permutations avoid $Q_k$,
    so $\abs{Av_{k-1}(Q_k)}=(k-1)!$.
    % 
    For $n\geq k$, $\pi$ is an $n$-permutation avoiding $Q_k$
    if and only if $n\in \pi_{[n-k+2,\, n]}$ 
    and the $(n-1)$-permutation obtained from removing $n$ from the permutation $\pi$ avoids $Q_k$.
    So $\abs{Av_{n}(Q_k)} = (k-1)\times \abs{Av_n (Q_{k-1})}$.
    The desired formula can then be easily obtained from this recursion via induction on $k$.
\end{proof}

\begin{definition}
    The \textit{permanent} of an $n\times n$ matrix $A_n=(a_{ij})_{i,j\in [n]}$ is 
    % denoted Perm($A_n$) and defined as 
    \[\text{Perm}(A_n):= \sum_{\pi\in S_n} \prod_{i=1}^n  a_{i j}.\]
\end{definition}

% \begin{lemma}
%     Using the recursive definition of the permanent, we can arrive at the following definition:
%     \[\text{Perm}(A_n)= \sum_{\pi\in S_n} \prod_{i=1}^n  a_{i \pi_i}\]
% \end{lemma}

% \begin{proof}
%     We claim that the sum above is equal to the permanent of the matrix $A_n$ defined above. 
%     That is, \[\sum_{\{\pi_1,\pi_2,\dots,\pi_n\}=[n]} \prod_{i=1}^n  a_{i \pi_i} = \text{Perm}(A_n)\]
%     We proceed by induction. The claim is obviously true for $n=1$.
%     So suppose the claim is true for some $k\geq 1$. 
%     Recall that the permanent of a matrix is defined similarly to the determinant of a matrix, except th at the $\pm$ signs in the sum. 
%     Let $M(j)$ denote the $M_{1,j}$ minor of the matrix $M$.
%     \begin{align*}
%         &\sum_{\{\pi_1,\pi_2,\dots,\pi_{k+1}\}=[k+1]} \left(\prod_{i=1}^{k+1}  a_{i \pi_i}\right) \\
%         &=\sum_{\pi_{k+1}=1}^{k+1} \left(\sum_{\{\pi_1,\pi_2,\dots,\pi_{k}\}=[k+1]\setminus \{\pi_{k+1}\}} \left(\prod_{i=1}^{k}  a_{i \pi_i}\right) a_{(k+1)\pi_{k+1}}\right) \\
%         &=\sum_{\pi_{k+1}=1}^{k+1}  \text{Perm} (A_{k+1}(\pi_{k+1})) \cdot a_{(k+1)\pi_{k+1}} \\
%         &= \text{Perm}(A_{k+1})
%     \end{align*}
% \end{proof}

The following proposition states a well-known property of permanents.
Its proof is similar to the proof for an analogous statement for deteminants
and is therefore omitted.

\begin{prop}[\textit{folklore}]
    The permanent of a matrix is invariant under arbitrary permutations of the rows and/or columns, 
    as well as transposition.
    That is, for all $n\times n$ matrices $M$ and $n$-permutation matrices $P$ and $Q$, 
    \begin{center}
        Perm$(M^T)=$ Perm$(M)=$ Perm$(PMQ)$.
    \end{center}
\end{prop}

\bigskip 
Observe that the permanent of the $n\times n$ matrix of ones is equal to $n!$ for all positive $n$,
and recall that there are $n!$ permutations in $S_n$.
One may ask whether there is a matrix associated with subsets of $S_n$, 
such that the problem of enumerating those subsets can be converted into 
computing a certain value of a matrix. 
This is in fact possible for certain subsets of $S_n$, as we shall see.

\begin{definition}
    Let $n\geq 1$. 
    Suppose $K$ is a set of restrictions that indicate whether 
    $i\mapsto j$ in an $n$-permutation is allowed for all $i,\, j\in [n]$.
    % 
    Then define $A_n^K=(a_{ij})$ as the binary $n\times n$ matrix 
    where, for all $i,\, j \in [n]$,
    the $ij$th element of $A_n^K$ is denoted $a_{ij}$ and is equal to 1 if and only if 
    having $i\mapsto j$ is allowed by $K$.
    % 
    We say that the matrix $A_n^K$ \textit{represents} $K$.
\end{definition}

\bigskip

\begin{lemma}[Percus (1971) \cite{percus}]\label{lemma:perm}
    Given a set of restrictions $K_n$ and matrix $A_n^K$ defined as above,
    the number of $n$-permutations that satisfy $K$ is the permanent of $A_n^K$. 
\end{lemma}
\begin{proof}
    Let $f$ be the function from $[n]$ to the set of subsets of $[n]$ such that $i\rightarrow \pi_i$ is allowed in a permutation if and only if $\pi_i\in f(i)$.
    Let $a_{ij}$ denote the $ij$th entry of $A_n^K$. 
    Then the number of $n$-permutations induced by $f$ is
    \begin{align*}
        \#\{\pi\in S_n \mid \pi_i\in f(i) \fa i\in [1,n]\}
        % &= \#\{\pi=\pi_1\pi_2\cdots \pi_n\mid (\pi_1,\pi_2,\dots,\pi_n)\in f(1)\times f(2)\times\cdots f(n), \quad \pi\in S_n\} \\
        % &= \#\{\pi=\pi_1\pi_2\cdots \pi_n\mid (\pi_1,\pi_2,\dots,\pi_n)\in f(1)\times f(2)\times\cdots f(n), \quad \{\pi_1,\pi_2,\dots,\pi_n\}=[n]\} \\
        % &=\#\{\pi=\pi_1 \pi_2\cdots \pi_n \mid a_{i \pi_i}=1 \quad \forall i\in [1,n], \quad \{\pi_1,\pi_2,\dots,\pi_n\}=[n]\}\\
        &=\#\{\pi\in S_n \mid a_{i \pi_i}=1 \fa i\in [1,n]\}\\
        &= \sum_{\pi\in S_n} a_{1 \pi_1} a_{2 \pi_2} \cdots a_{n \pi_n} \\
        % &= \sum_{\{\pi_1,\pi_2,\dots,\pi_n\}=[n]} a_{1 \pi_1} a_{2 \pi_2} \cdots a_{n \pi_n} \\
        &= \sum_{\pi\in S_n} \prod_{i=1}^n  a_{i \pi_i} \\
        % &= \sum_{\{\pi_1,\pi_2,\dots,\pi_n\}=[n]} \prod_{i=1}^n  a_{i \pi_i} 
        &= \text{Perm}(A).
    \end{align*} 
    % (Coincidentally, permutations and permanent both start with 'perm'.)
\end{proof}

We can now apply the theorem to the enumeration of the avoidance set of $Q_k$ for all $k\geq 1$:
% 
Observe that for all $n$, an $n$-permutation $\pi$ avoids $Q_k$ if and only if 
$t\in \pi_{[t-k+2,n]}$ for all $t\in [k,n]$.
% That is, 
% \[\pi_i=t\iff i\in [t-k+2,n] \iff i\geq t-k+2\]
Therefore we have the following theorem:

\begin{theorem}\label{thm:Q-matrix}
    The $n$-permutations that avoid $Q_k$ are represented by the binary $n\times n$ matrix
    $A_n=(a_{ij})_{i,j\in [n]}$ where $a_{ij}=1$ if and only if $i\geq j-k+2$.
\end{theorem}

Since the permanent of $A_n$ 
is exactly $(k-1)! (k-1)^{n-k+1}$ for all $n\geq k$,
the theorem above proves the enumeration of the avoidance set of $R_k$ for all $n$.

% \newpage
\section{Juggling sequences}\label{sect:juggling}
% (Problem \#13)
Suppose we have $b$ balls and a binary vector $\sigma=(\sigma_1,\sigma_2,\dots,\sigma_n)\in \{0,1\}^n$ for some integer $n\geq b$.
Given a reference time, 
we can throw one ball at the $i$th second 
such that it lands in our hand again after 
exactly $t_i$ seconds (that is, $i+t_i$ seconds after the reference time) 
where $t_i$ is a positive integer  
for all $i\in [n]$, if $\sigma_i=1$.
Otherwise, we put $t_i:=0$.
We say that the $n$-tuple of non-negative integers $T=(t_1,t_2,\dots,t_n)$ is 
a \textit{juggling sequence of period $n$ and \textit{state} $\sigma$}
if and only if 
no two balls land in our hand at the same time, 
and our sequence of throws is infinitely repeatable.
That is, the following two conditions hold:
\begin{itemize}
    \item $i+t_i\equiv j+t_j\pmod{n}$ if and only if $i=j$ for all $i,j\in [n]$,
    % That is, the map $i\rightarrow i+t_i \pmod{n}$ from $[n]$ to itself is injective.
    \item $\sigma_i=1$ if and only if there is some $j\in [n]$ such that $i\equiv j+t_j\pmod{n}$.
\end{itemize}
% 
We say that $\sigma$ is a \textit{ground state} if and only if 
$\sigma_i=1$ if $i\in [b]$ and $\sigma_i=0$ otherwise.
% 
We refer the reader to \cite{juggling-drops} and \cite{chung-graham} for diagrams and further analysis on juggling sequences.\\

Gao and Kitaev \cite{gao-kitaev-2019} observed that the avoidance set $Av_n(Q_5)$ and the 
number of ground-state 3-ball juggling sequences of period $n$
are enumerated by the same OEIS sequence \href{https://oeis.org/A084509}{A084509}.
We will prove a natural bijection between a generalization of these two combinatorial objects,
which is inspired by the proof of Theorem 1 of \cite{chung-graham}, restated below:

\begin{theorem}\label{thm:juggling-enum}
    The number of ground state juggling sequences of period $n$ using $b$ balls ($n\geq b$) is 
    \[J(n,b)=\begin{cases}
        (b+1)^{n-b} b!&\text{if }n\geq b,\\
        n!&\text{otherwise}.
    \end{cases}\]
\end{theorem}

\begin{proof}
    Observe that $T=(t_1,t_2,\dots,t_n)$ is a conforming juggling sequence if and only if 
    every ball that is thrown lands exactly at some time $t$ seconds after the reference time
    where $t\in [n+1,n+b]$, and no two balls land on the same second.
    That is, $\{t_i+i\mid i\in [b]\} = [n+1,n+b]$.
    Since $t_i=0$ for $i\in [n]\setminus [b]$, 
    the condition is equivalent to requiring 
    % $\{t_i+i\mid i\in [n]\} = [b+1,b+n]$.
    $\{t_i+i-b\mid i\in [n]\} = [n]$.
    That is, the bijection $i\mapsto t_i+i-b$ defines a permutation on the set $[n]$,
    with the additional condition that $t_i\geq 0$ for all $i\in [n]$.
    Thus every conforming juggling sequence $T$ can be associated uniquely to a permutation $\pi$
    where $\pi_i=t_i+i-b$.
    The number of such permutations is then exactly the permanent of the matrix 
    $M$ where its $ij$th entry is 1 if and only if $j-i+b\geq 0$, by Lemma \ref{lemma:perm}.
    Computing the permanent of $M$ yields the formula assigned to $J(n,b)$.
\end{proof}

\begin{table}[!htbp]
    \centering
    \begin{tabular}{|c|c|c|c|c|}
        \hline
        \thead{State,\\ period,\\ \# balls,\\\# juggling\\sequences} & \thead{Juggling\\sequences\\$T=(t_1,\dots,t_n)$} & \thead{$(t_1+1,\dots,t_n+n)$\\$=(\sigma_1,\dots,\sigma_n)$\\$+(b,\dots,b)$} & \thead{$\sigma\inv=\pi$\\$\in Av_n(Q_{b+2})$;\\$\sigma_i = t_i+i-b$} & \thead{Matrix\\transpose\\$M^t=M_{n,b}^t$} \\\hline
        \makecell{$\sigma=(1,0)$;\\$n=2$;\\ $b=1$;\\$J(n,b)=2$}      & \makecell{(1,1)\\(2,0)}  & \makecell{(2,3)\\(3,2)} & \makecell{$12\inv=12$\\$21\inv=21$} & $\left(\begin{array}{cc} 1&1\\1&1 \end{array}\right)$  \\
        \makecell{$\sigma=(1,0,0)$;\\$n=3$;\\ $b=1$;\\$J(n,b)=4$}    & \makecell{(1,1,1)\\(3,0,0)\\(2,0,1)\\(1,2,0)}  & \makecell{(2,3,4)\\(4,2,3)\\(3,2,4)\\(2,4,3)} & \makecell{$123\inv=123$\\$312\inv=231$\\$213\inv=213$\\$132\inv=132$} & $\left(\begin{array}{ccc} 0&1&1 \\ 1&1&1 \\ 1&1&1 \end{array}\right)$ \\
        \makecell{$\sigma=(1,0,0,0)$;\\$n=4$;\\ $b=1$,\\$J(n,b)=8$}  & \makecell{(1,1,1,1)\\(3,0,0,1)\\(2,0,1,1)\\(1,2,0,1)\\(2,0,2,0)\\(4,0,0,0)\\(1,3,0,0)\\(1,1,2,0)}  & \makecell{(2,3,4,5)\\(4,2,3,5)\\(3,2,4,5)\\(2,4,3,5)\\(3,2,5,4)\\(5,2,3,4)\\(2,5,3,4)\\(2,3,5,4)} & \makecell{$1234\inv=1234$\\$3124\inv=2314$\\$2134\inv=2134$\\$1324\inv=1324$\\$2143\inv=2143$\\$4123\inv=2341$\\$1423\inv=1342$\\$1243\inv=1243$} & $\left(\begin{array}{cccc} 0&0&1&1 \\0&1&1&1 \\ 1&1&1&1 \\ 1&1&1&1 \end{array}\right)$ \\
        \makecell{$\sigma=(1,1)$;\\$n=2$;\\ $b=2$;\\$J(n,b)=2$}      & \makecell{(2,2)\\(3,1)}  & \makecell{(3,4)\\(4,3)} & \makecell{$12\inv=12$\\$21\inv=21$} & $\left(\begin{array}{cc} 1&1\\1&1 \end{array}\right)$  \\
        \makecell{$\sigma=(1,1,0)$;\\$n=3$;\\ $b=2$;\\$J(n,b)=6$}    & \makecell{(2,2,2)\\(2,3,1)\\(3,1,2)\\(3,3,0)\\(4,1,1)\\(4,2,0)} & \makecell{(3,4,5)\\(3,5,4)\\(4,3,5)\\(4,5,3)\\(5,3,4)\\(5,4,3)} & \makecell{$123\inv=123$\\$132\inv=132$\\$213\inv=213$\\$231\inv=312$\\$312\inv=231$\\$321\inv=321$} & $\left(\begin{array}{ccc} 1&1&1 \\ 1&1&1 \\ 1&1&1 \end{array}\right)$ \\\hline
    \end{tabular}
    \label{table:avQ-juggling}
    \caption{Sample values for juggling sequences and their images under $\theta$ for small $n$ and $b$. 
    Note that we are using the permutation matrix notation system
    where rows are numbered from bottom to top in increasing order.}
\end{table}

The desired bijection can then be easily obtained by realizing that 
the permutations mentioned in the proof of Theorem \ref{thm:juggling-enum}
are exactly the permutation inverses of those avoiding $Q_{b+2}$, 
as is carefully fleshed out in the following theorem:

\begin{theorem}
    Let $\theta$ be a function from the set of ground state juggling sequences of period $n$ using $b$ balls to $Av_n(Q_{b+2})$
    given by $\theta((t_1,t_2,\dots,t_n))=\pi$ 
    where $\pi_{t_i+i-b}=i$ for all $i\in [n]$.
    Then $\theta$ is a bijection.
\end{theorem}
\begin{proof}
    Since $t_i$ is nonnegative for all $i\in [n]$, we have 
    \[\pi_{t_i+i-b}=i 
    \iff \pi_{i}=i+b-t_i 
    % \iff \pi_{i}-b=i-t_i 
    \iff i \geq \pi_i-b 
    = \pi_i - i - (b+2) + 2.
    \]
    Therefore, for a given $n,b$, the matrix $M$ defined in Chung and Graham's paper 
    is exactly the transpose of the matrix that represents the avoidance of $Q_{b+2}$ as defined in Theorem \ref{thm:Q-matrix}. 
    So the codomain of $\theta$ is indeed $Av_n(Q_{b+2})$.
    % 
    By Theorem \ref{thm:avQk},
    the number of ground state $b$-ball juggling sequences of period $n$
    is equal to 
    the size of $Av_n(Q_{b+2})$
    since 
    \[Av_n(Q_{b+2}) = \begin{cases}
        (b+1)!\times (b+1)^{n-b-1} = (b+1)^{n-b}b! &\text{if }n\geq b+2,\\
        n!&\text{otherwise.}
    \end{cases}\]

    Therefore, since $\theta$ is injective,
    it must also be bijective.
    % The inverse of $\theta$ is given by
    %     \[\theta\inv(\pi) = (\pi(1)+b-1, \pi(2)+b-2, \dots, \pi(n)+b-n).\]
\end{proof}

\noindent We refer the reader to Table \ref{table:avQ-juggling} for sample values for small $n$ and $b$.

% \todo{Generalize to: badminton sequences (machine production), mancala game}

\section{Shrub forests of \texorpdfstring{$n$}{n} heaps}\label{sect:shrub}
\begin{definition}
    Let $\mathcal{P}_{3n}$ denote the set of permutations of length $3n$
    that avoid the patterns 231, 312 and 321 
    and satisfies $\pi_{3i+1} < \pi_{3i+2}$ and $\pi_{3i+1} < \pi_{3i+3}$ for all $i\in [n-1]$.
\end{definition}

\begin{remark}
    $\mathcal{P}_{3n}$ is also known as the set of \textit{2-ary shrub forests of $n$ heaps} avoiding the patterns 231, 312 and 321.
    % and has size $2\times 3^{n-1}$ (the sequence A025192). 
    % \footnote{D. Bevan, D. Levin, P. Nugent, J. Pantone, L. Pudwell, M. Riehl, and M. Tlachac. Pattern avoidance in forests of binary shrubs. Discr. Math. Theor. Comp. Sci., 18(2): 2016.}
\end{remark}

\bigskip
\noindent Gao and Kitaev \cite{gao-kitaev-2019} observed that $Av_n(Q_4)$ and the 
$\mathcal{P}_{3n}$
are enumerated by the same OEIS sequence \href{https://oeis.org/A025192}{A025192}.
We will show a natural bijection between these two sets for all $n\geq 1$.

\begin{theorem} \label{thm:avQ-P}
    Let
    \[P_{3n}=\begin{cases}
        \{123,\, 132\} &\text{if }n=1,\\ 
        \{1 \oplus \tau \oplus \pi_{[2,3n-3]}\mid \tau\in \{123,\, 132,\, 213\}\text{ and } \pi\in \mathcal{P}_{3n-3}\} &\text{if }n\geq 2.
    \end{cases}.\]
    Then $P_{3n}=\mathcal{P}_{3n}$.
\end{theorem}

\begin{proof}
    It is easy to see that $\mathcal{P}_3=\{123,\, 132\}$.
    Let $\sigma:=1 \oplus \tau \oplus \pi_{[2,3n-3]}$
    for some $\tau\in \{123,\, 132,\, 213\}$
    and $\pi\in \mathcal{P}_{3n-3}$. 
    It is clear that
    \[\abs{\sigma}=\abs{1 \oplus \tau \oplus \pi_{[2,3n]}}=1+3+\abs{\pi_{[2,3n]}}=4+3n-3-1=3n,\]
    so $\sigma$ is a $3n$-permutation.
    % 
    We know that $\pi\in \mathcal{P}_{3n-3}$, 
    so $\sigma_{3i+1} < \sigma_{3i+2},\, \sigma_{3i+3}$ for $i=3,4,\dots,n-1$.
    % 
    By the definition of the direct sum $\oplus$
    we have $\sigma_1 < \sigma_{[2,4]} < \sigma_{[5,3n]}$, 
    so $\sigma_{3i+1} < \sigma_{3i+2}$ and $\sigma_{3i+1} < \sigma_{3i+3}$ for $i=1$ and 2 as well.
    Since $\pi_{[2,3n-3]}$
    and $\tau=S_3\setminus \{231,\,312,\,321\}$
    both avoid 231, 312 and 321,
    so do the factors $\sigma_{[1,4]}$ and $\sigma_{[5,3n]}$.
    It is easy to see (by the definition of $\oplus$)
    that none of the forbidden patterns may span across $\sigma_{[1,4]}$ and $\sigma_{[5,3n]}$.
    Therefore $\sigma\in \mathcal{P}_{3n}$.\\

    We claim that $\abs{P_{3n}}=2\times 3^{n-1}$ for all $n\geq 1$.
    Indeed, $\abs{P_3}=2=2\times 3^0$.
    Suppose this is true for some $k\geq 2$.
    Since $\pi_1=1$ for all $\pi\in P_{3k}$,
    we must have that $\pi_{[2,3k]}$ is distinct for each $\pi\in P_{3k}$.
    Therefore $\abs{P_{3k+3}}=3\times \abs{P_{3k}}=2\times 3^{n-1}$,
    and the claim is true for all $n$.
    % 
    Since $P_{3n}\subseteq \mathcal{P}_{3n}$ and $\abs{P_{3n}}=\abs{\mathcal{P}}_{3n}$, they must be equal for all $n\geq 1$. 
\end{proof}

\begin{theorem}\label{thm:avQ4-bij}
    % There exists a natural bijection between $Av(Q_4)$ and the $3n$-permutations
    % $Av(231, 312, 321)$ where $\pi_{3i+1} < \pi_{3i+2}, \pi_{3i+3}$ $1\leq i<n$, given by
    Let $\theta: Av_n(Q_4)\rightarrow \mathcal{P}_{3n-3}$ be given by 
    \begin{align*}
        \theta(12)&=123, \quad \theta(21)=132, \quad \text{and for }\abs{\pi}=n\geq 3,\\
        % \theta\left(\pi_{[1,n-1]}\,n\right ) &= 1 \oplus 132 \oplus \theta\left(\pi_{[1,n-1]}\right)_{[2,3n]},\\
        % \theta\left(\pi_{[1,n-2]}\,n\,\pi_n\right ) &= 1 \oplus 123 \oplus \theta\left(\pi_{[1,n-2]}\,n\,\pi_n\right )_{[2,3n]},\\
        % \theta\left(\pi_{[1,n-3]}\,n\,\pi_{[n-1,n]}\right ) &= 1 \oplus 213 \oplus \theta\left(\pi_{[1,n-3}\,n\,\pi_{[n-1,n]} \right )_{[2,3n]}.
        \theta(\pi) &= 
        \begin{cases}
            1 \oplus 132 \oplus \theta\left(\pi_{[1,n-1]}\right)_{[2,3n]}, &\text{if }\pi_n=n,\\
            1 \oplus 123 \oplus \theta\left(\pi_{[1,n-2]}\,\pi_n\right )_{[2,3n]},&\text{if }\pi_{n-1}=n,\\
            1 \oplus 213 \oplus \theta\left(\pi_{[1,n-3]}\,\pi_{[n-1,n]} \right )_{[2,3n]} &\text{if }\pi_{n-2}=n.
        \end{cases}
    \end{align*}
    Then $\theta$ is a bijection.
\end{theorem}

\begin{figure}[!htbp]
    \centering
    \tikzfig{./}{Q4}
    \caption{The POP $Q_4$}
    \label{fig:Q4}
\end{figure}

\begin{proof}
    Recall that $k$ can only be $\pi_{n}$, $\pi_{n-1}$ or $\pi_{n-2}$ by the proof of Theorem \ref{thm:Q-matrix}.
    So $\theta$ is defined on $Av_n(Q_4)$ for $n\geq 3$.
    % 
    Moreover, $\theta(Av_n(Q_4))\subseteq S_{3n-3}$ for all $n\geq 2$.
    The base case is: $\abs{\theta(12)}=\abs{\theta(21)}=3$.
    Suppose $\theta(Av_{k-1}(Q_4))\subseteq S_{3k-6}$ for some $k\geq 3$.
    Then if $\pi\in Av_k(Q_4)$,
    we have \[\abs{\theta(\pi)}=1+3+(3(k-1)-2+1)=3k=3(k+1)-3.\]
    So $\theta$ maps into $S_{3n-3}$ for all $n$. 
    % \rremark{is this para necessary?}
    % 
    Obviously $\theta(Av_2(Q_4))=\mathcal{P}_{3}$.
    Suppose $\theta(Av_{k-1}(Q_4))\subseteq \mathcal{P}_{3k-6}$ for some $k\geq 2$.
    Then by Theorem \ref{thm:avQ-P}, 
    $\theta$ indeed maps $Av_k(Q_4)$ into $\mathcal{P}_{3k-3}$.
    So $\theta(Av_{n}(Q_4))\subseteq \mathcal{P}_{3n-3}$ for all $n\geq 2$ by induction.\\

    Finally, it is clear from the definition of $\theta$ that it is injective.
    By Theorem \ref{thm:avQk}, $\abs{Av_n(Q_4)} = 3!\times 3^{n-3} = 2\times 3^{n-2}$ for $n\geq k$.
    Since $\abs{Av_n(Q_4)}=\abs{\mathcal{P}_{3n-3}}$ for all $n$,
    the map $\theta$ must be surjective as well.
\end{proof}

\begin{table}[!htbp]
    \centering 
    \begin{tabular}{|c|c|c|}
        \hline
        $n$ & $Av_n(Q_4)$ & $\mathcal{P}_{3n-3}$ \\\hline 
        2   & \makecell{12\\21} & \makecell{123\\132} \\
        3   & \makecell{123\\213\\132\\231\\312\\321}   & \makecell{124356\\124365\\123456\\123465\\132456\\132465} \\
        \hline
    \end{tabular}
    \label{}
    \caption{Sample values of $Av_n(Q_4)$ and their images under $\theta$ in $\mathcal{P}_{3n}$}
    % \quad
    % \begin{tabular}{ccc}
    %     $n$ & $Av_n(Q_4)$ & $\mathcal{P}_{3n-3}$ \\
    %     4   & \makecell{1234\\1243\\1423\\2134\\2143\\2413\\1324\\1342\\1432\\2314\\2341\\2431\\3124\\3142\\3412\\3214\\3241\\3421} 
    %         & \makecell{} \\
    % \end{tabular}
\end{table}

We note that our bijection is easily generalized:

\begin{definition}
    Let $Q_{k,j}$ be the POP of size $k$ where $i<j$ for all $i,j\in [k]$ where $i\neq j$,
    as illustrated in Figure \ref{fig:Qkj}.
    \begin{figure}[!htbp]
        \centering
        \tikzfig{./}{Qkj}
        \caption{The POP $Q_{k,j}$, where $\{j,i_1,i_2,\dots,i_{k-1}\}=[k]$}
        \label{fig:Qkj}
    \end{figure}
\end{definition}

\begin{theorem}
    There is a natural bijection between $Av_n(Q_{4,j})$ and $\mathcal{P}_{3n-3}$ for all $1\leq j\leq 4$.
\end{theorem}

\begin{proof}
    Due to the symmetry of $Q_{k,j}$ and the fact that $Q_k=Q_{k,1}$,
    it is not hard to see that $\pi\in Av_n(Q_{k})$
    if and only if $\pi_j\, \pi_{[2,j-1]} \,\pi_1 \,\pi_{[j+1,n]}\in Av_n(Q_{k,j})$.
    % 
    We can then obtain a bijection from between $Av_n(Q_{4,j})$ and $\mathcal{P}_{3n-3}$ 
    for all $1\leq j\leq 4$
    by composing $\theta$ from Theorem \ref{thm:avQ4-bij}
    with the bijection $\pi \mapsto \pi_j \,\pi_{[2,j-1]}\, \pi_1 \,\pi_{[j+1,n]}$.
\end{proof}

% \newpage
% \section{Skew-sum decomposable permutations}
% This is a separate observation that might prove useful in future: \\

% Let $Q_k=q_1 q_2\dots q_k$ be the POP where $q_1>q_i$ for $i=2,3,\dots,k$ . 
% Then the number of skew-sum decomposable $n$-permutations avoiding $Q_k$ for $k\geq 3$ is 
% \[\sum_{i=1}^{k-2}Av_{n-i}(Q_{k-i})i!\]
% \begin{proof}
%     A permutation $\pi=21[\alpha_1,\alpha_2]$ avoids $Q_k$ if and only if 
%     $1\leq \abs{\alpha_2}=:i\leq k-2$, and $\alpha_1\in Av_{n-i}(Q_{k-i})$.
% \end{proof}

% \begin{corollary}
%     The number of skew-sum decomposable $n$-permutations avoiding $Q_k$ for $k\geq 3$ is 
%     \[(k-1)!(k-1)^{n-(k-1)}-\left(\sum_{i=1}^{k-2}Av_{n-i}(Q_{k-i})i!\right)\]
% \end{corollary}
% \begin{proof}
%     This is immediate from Theorem 2 in Gao-Kitaev's 2019 paper.
% \end{proof}
\end{document}