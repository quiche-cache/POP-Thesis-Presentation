% !TEX root = ../../main.tex
\documentclass[../main.tex]{subfiles}
%
\begin{document}


Albert and Atkinson \cite{albert-atkinson} proved that 
a permutation class with only finitely many simple permutations has a readily computable algebraic generating function
and has a finite basis.
So far, we have been dealing mostly with avoidance sets that contain only finitely many simple permutations.
In this chapter, we will show an example of a permutation class with a finite basis 
that contains infinitely many simple permutations 
as well as construct an algorithm that allows us to obtain the entire set recursively.

\begin{definition}
    Let $P$ be a pattern, a set of patterns, or a POP.
    Denote $Av_n^S(P)$ as the set of simple $n$-permutations avoiding $P$.
\end{definition}

\begin{theorem}\label{thm:fib}
    The set of simple permutations avoiding $P:=\{2413,\,3412,\,3421\}$ is enumerated by a translate of the well-known Fibonacci sequence.
    Specifically, if $n\geq 3$, then 
    $\abs{Av_n^S(P)}=F(n-3)$
    where $F(0)=0$, $F(1)=1$ and $F(n)=F(n-1)+F(n-2)$ for all $n\geq 3$. 
\end{theorem} 

\section{Partitioning the simple permutations into 3 sets}

\noindent In order to enumerate the set of simple permutations avoid 2413, 3412 and 3421,
we will identify the types of permutations that appear in it. 

\begin{lemma}\label{lemma:fib_structure}
    Let $\pi $ be a simple $n$-permutation that avoids $P$ for some $n\geq 4$.
    Then $\pi_{n-1}=n$ and $\pi_1=n-1$.
    % \item $\pi_2=n-3$ or 1.
\end{lemma}

\begin{proof}
    We first prove that $\pi_{n-1}=n$.
    Let $j$ and $k$ be in $[n]$ such that $\pi_k=n$
    and $\pi_j := \max(\{\pi_1,\, \pi_2,\, \dots,\, \pi_{k-1]}\})$.
    Suppose there exists some $\ell>k$ such that $\pi_\ell<\pi_{j}$.
    Note that $\text{red}(\pi_j\, \pi_k\, \pi_{\ell_1}\, \pi_{\ell_2})$
    would be 3412 or 3421 if $\ell_1 < \ell_2$ and 
    both $\ell_1$ and $\ell_2$ both satisfied conditions for $\ell$.
    So $\ell$ must be unique.
    If $\ell\neq n$ then $\pi_j<\pi_n$
    so $\pi_{j}\,\pi_k\,\pi_\ell\,\pi_n=\pi_{j}\, n\,\pi_\ell\,\pi_n$
    reduces to 2413.
    So $\ell=n$. 
    We know that if $i>k$ and $i\neq \ell$ then $\pi_i>\pi_{j}$,
    which implies that $\pi_{[k,n-1]}$ is an interval.
    Since $\pi$ is simple and $k>1$, we must have $k=n-1$.  \\
    
    Next, we prove that $\pi_1=n-1$.
    Let $\pi$ be represented as the string of factors $\alpha\, (n-1)\, \beta\, n\, k$ where $k\in [n-2]$.
    We know that $\beta$ cannot be empty, since then $(n-1)n$ would be an nontrivial interval in $\pi$.
    Suppose $\alpha$ is not empty.
    Note that the subsequence $\hat{\alpha}\, \hat{\beta}\, k$ 
    must reduce to 123 or 132 or 231
    for all $\hat{\alpha}\in \alpha$ and $\hat{\beta}\in \beta$ by elimination 
    (since $n-1$ is larger than the three points and $\hat{\alpha}\,(n-1)\, \hat{\beta}\, k$ cannot reduce to 2413, 3412 or 3421).
    The set of patterns $\{123,\, 132,\, 231\}$ is equivalent to the POP of size three where $1<2$,
    so the above observation implies that $\alpha < \beta$. 
    If $\alpha > k$ then $\pi$ would be skew sum decomposable,
    while if $\alpha < k$ then $\pi$ would be sum decomposable, 
    both of which contradict the simpleness of $\pi$.
    But this implies that $\beta > k$ which makes $(n-1)\, \beta\, n$ a nontrivial interval of $\pi$.
    So $\alpha$ must be empty, meaning that $\pi_1=n-1$.
\end{proof}
    
\begin{lemma}\label{lemma:fib_structure2}
    Let $\pi $ be a simple $n$-permutation that avoids $P$ where $n\geq 5$.
    Then we have exactly 3 cases:
    \begin{enumerate}
        \item $\pi_2=1$ and $\pi_3=n-2$,
        \item $\pi_2=n-3$ and $\pi_{n-2}=n-2$, or 
        \item $\pi_2=1$, $\pi_3=n-3$ and $\pi_{n-2}=n-2$.
    \end{enumerate}
\end{lemma}

\begin{proof}
    We know from Lemma \ref{lemma:fib_structure} that $\pi_{n-1}=n$ and $\pi_1=n-1$.
    Write $\pi$ as the string of factors $(n-1)\,\gamma \, n\, k$ where $k$ has length 1. 
    If $k=n-2$ then $\gamma$ is an interval of length $n-3\geq 2$. So $k\in [n-3]$ and $n-2\in \gamma$.
    Let $s:=\abs{\gamma}$ and $t\in [s]$ such that $\gamma_t=n-2$.\\
    
    Suppose $t<s$. We want to show that $t=2$ and $\gamma_1=1$, which will satisfy case a.
    If $t=1$ then $\pi_{[1,2]}=(n-1)\, (n-2)$ is a nontrivial interval, so $t>1$. 
    So $t\in [2,s-1]$.
    Note that $\gamma_i\, (n-2)\, \gamma_j\, k$ reduces to 1423, 1432 or 2431 for all $1\leq i<t<j\leq s$ 
    by elimination since $n-2$ is the largest of the four points,
    and the subsequence cannot reduce to 2413, 3412 or 3421.
    This observation implies that $\gamma_{[1,t-1]}<\gamma_{[t+1,s-1]}$.
    So $\gamma_{[1,t-1]}<\gamma_{[t,s]}$.
    If $k<\gamma_{[t,s]}$ then $\gamma_{[t,s]}$ is a nontrivial interval.
    So $\gamma_{[1,t-1]} < k$ which means that $\gamma_{[1,t-1]}$ is an interval.
    By the simpleness of $\pi$ we have $t=2$. 
    Since $\gamma_{[1,t-1]}=\gamma_1$ is smaller than all the other points, $\gamma_1=1$.\\

    Now suppose that $t=s$. That is, $\pi_{n-2}=n-2$. 
    If $k=n-3$ then $\gamma$ is an interval of length 1.
    This is only possible if $\pi=41352$, which satisfies case a.
    So if $n\geq 6$ then $n-3$ is in $\gamma$ instead of $k$. 
    Let $n\geq 6$ and $m\in [s]$ such that $\gamma_m=n-3$. 
    If $m=1$ then $\pi$ satisfies case b.
    % 
    So suppose $m>1$.
    We know that $m\neq s-1$ since that would mean that $\gamma_{[s-1,s]}=(n-3)\,(n-2)$ is a nontrivial interval in $\pi$.
    So $m\in [2,s-2]$.
    Note that $\gamma_i\, (n-3)\, \gamma_j\, k$ reduces to 1423, 1432 or 2431 for all $1\leq i<m<j\leq s-1$, 
    since $n-3$ is the largest of the four points
    and the subsequence cannot reduce to 2413, 3412 or 3421. 
    This observation implies that $\gamma_{[1,m-1]} <\gamma_{[m+1,s-1]}$.
    So $\gamma_{[1,m-1]} <\gamma_{[m,s]} < n-1$.
    If $k<\gamma_{[m,s]}$ then $\gamma_{[m,s]}$ is a nontrivial interval.
    So $\gamma_{[1,m-1]} < k$ which means that $\gamma_{[1,m-1]}$ is an interval.
    By the simpleness of $\pi$ we have $m=2$. 
    Since $\gamma_{[1,m-1]}=\gamma_1$ is smaller than all the other points, $\gamma_1=1$.\\
\end{proof}

% \begin{corollary}\label{cor:fib-pin}
%     Suppose $\pi \in Av_n^S(P)$. Then the following hold:
%     \begin{enumerate}[(a)]
%         \item If $n\geq 5$ then $\pi_n \in [2,n-3]$.
%         \item If $n\geq 7$ and $\pi_2=1$, then $\pi_n\in [3,n-3]$.
%     \end{enumerate} 
% \end{corollary}
% \begin{proof}
%     Let $\pi \in Av_n^S(P)$.
%     \begin{enumerate} [(a)]
%         \item By Lemma \ref{lemma:fib_structure}, $\pi_1=n-1$ and $\pi_{n-1}=n$.
%         So $\pi_n\in [n-2]$.
%         % 
%         Suppose $\pi_n=n-2$.
%         Then $\pi_{[2,n-2]}$ would be the interval $[n-3]$,
%         which has length at least 2 (for $n\geq 5$),
%         contradicting the assumption that $\pi$ is simple. 
%         So $\pi_n<n-2$.
%         % 
%         We cannot have $\pi_n=1$ since otherwise $\pi$ would be skew sum decomposable,
%         so $\pi_n>1$. 
        
%         \item Now suppose $n\geq 7$, $\pi_2=1$ 
%         and $\pi_n=2$. 
%         Then $\pi_{[3,n-2]}$ would be the interval $[3,n-3]$,
%         which has length at least 2 (for $n\geq 7$),
%         So $\pi_n>2$. 
%         From part (a), we must have $\pi_n\in [3,n-3]$. 
%     \end{enumerate}
% \end{proof}

% \begin{lemma}\label{lemma:pi3}
%     Suppose $\pi$ is a simple $n$-permutation avoiding $P$.
%     If $n\geq 6$ and $\pi_2=1$, then $\pi_3=n-2$ or $n-3$. 
%     Moreover, if $\pi_3=n-3$, then $\pi_{n-2}=n-2$.
% \end{lemma}
% \begin{proof}
%     We wish to prove that if $n\geq 6$ and $\pi_2=1$, then one of the following holds:
%     \begin{enumerate}[(a)]
%         \item $\pi_3=n-2$, or
%         \item $\pi_3=n-3$ and $\pi_{n-2}=n-2$.
%     \end{enumerate}
%     First suppose that \[\pi=(n-1)\,1\,\rho\,(n-2)\,\tau\, n\,k,\] 
%     where $\rho$ and $\tau$ are factors of $\pi$ and $k\in [n]$.
%     It suffices to show that either $\rho$ is empty, or $\tau$ is empty and $\rho_1=n-3$.\\
    
%     Suppose $\rho$ and $\tau$ are both not empty.
%     If there exists some $\hat{\rho}\in \rho$ and $\hat{\tau}\in \tau$,
%     such that $\hat{\rho}> \hat{\tau}$,
%     then the subsequence 
%     $\hat{\rho}\,(n-2)\, \hat{\tau}$ 
%     must reduce to the pattern $231$.
%     This implies that the subsequence 
%     $\hat{\rho}\,(n-2)\, \hat{\tau}\,k$ 
%     reduces to the pattern $2413, 3412$ or $3421$, 
%     which is impossible.
%     So $\rho<\tau$, and $\tau$ must contain the number $n-3$.
%     This means that $\tau$ must have length greater than 1, 
%     so that $(n-2)\, \tau$ is not an interval of length in $[2,n-2]$.
%     Since $\pi$ is simple, 
%     we cannot have $\rho<k$ nor $k<\tau$ 
%     since either way would result in the inclusion of an interval of length in $[2,n-2]$.
%     This is impossible, so either $\rho$ or $\tau$ must be empty.\\
    
%     Now suppose that $\tau$ is empty, so $\pi_{n-2}=n-2$.
%     Then $\abs{\rho}=n-4\geq 2$. 
%     So if $n-4=2$, then $\rho_1=n-3$.
%     Now suppose $n-4\geq 3$ and say $\rho_i=n-3$ for some $i\in [n-4]$
%     and $\widehat{\rho_{[1,i-1]}}\,(n-3)\,\widehat{\rho_{[i+1,n-4]}} $
%     is the pattern $231$
%     for some $\widehat{\rho_{[1,i-1]}}\in \rho_{[1,i-1]}$ and $\widehat{\rho_{[i+1,n-4]}}\in \rho_{[i+1,n-4]}$,
%     then \[\widehat{\rho_{[1,i-1]}}\,(n-3)\,\widehat{\rho_{[i+1,n-4]}}\,k\]
%     is the pattern 2413, 3412 or 3421, which is forbidden.
%     So $\rho_{[1,i-1]}<\rho_{[i+1,n-4]}$.
%     Since $\pi$ is simple, 
%     we cannot have $\rho_{[1,i-1]}<k$ nor $k<\rho_{[i+1,n-4]}$,  
%     since either way would result in the inclusion of an interval of length in $[2,n-2]$.
%     This is impossible, so either $i=1$ or $i=n-4$. 
%     If $i=n-4$, then $\pi$ would contain $(n-3)\,(n-2)$ as a factor which is impossible.
%     So $\rho_1=n-3$.
% \end{proof}

% \begin{lemma}
%     If $n\geq 6$ and $\pi_2=1$, then $\pi_3=n-2$ or $n-3$. 
% \end{lemma}
% \begin{proof}
%     Suppose not. Then let $\pi_3=m$, and let $\pi_i=n-2$, $\pi_j=n-3$
%     for some $i,\, j\in [n]$.
%     Since $\pi$ is simple, we know that $i$ and $j$ cannot be adjacent.

%     \begin{itemize}
%         \item Case 1: $i<j-1$.
        
%         \item Case 2: $j<i-1$.
%         If $i\leq n-3$, 
%         then since $\pi_1=n-1$ and $\pi_{n-1}=n$ by Lemma \ref{lemma:fib_structure},
%         we must have $\pi_{n-2}$, $\pi_{n}\in [n-4]$.
%         But this means that 
%         $\pi_j\, \pi_i\, \pi_{j}\, \pi_i$
%         must reduce to 3412 or 3421, 
%         which is impossible.
%         So $i\geq n-2$.
%     \end{itemize}
% \end{proof}

% \begin{corollary}\label{cor:fib-n-2}
%     Suppose $\pi$ is a simple $n$-permutation avoiding $P$.
%     For $n\geq 5$,
%     either $\pi_3=n-2$ or $\pi_{n-2}=n-2$. 
% \end{corollary}
% \begin{proof}
%     Suppose $\pi_3\neq n-2$, $\pi_i=n-3$ and $\pi_j=n-2$.
%     Then by Lemma \ref{lemma:fib_structure} and \ref{lemma:pi3},
%     we have $i=2$ or 3.
%     Moreover, for all $\ell\in \{i+1, i+2, \dots, n-2\}\setminus \{j\}$,
%     the term $\pi_{\ell}$ can be at most $n-4$,
%     since it cannot be $n-3$, $n-2$, $n-1$ or $n$.
%     We also know that $\pi_n\leq n-3$ by part (a) of Corollary \ref{cor:fib-pin}.
% % 
%     So if $j<n-2$, then 
%     \[\text{red}(\pi_i\, \pi_{j}\, \pi_{j+1}\pi_n) 
%     = \text{red}((n-3)\, (n-2)\, \pi_{j+1}\pi_n)
%     = 3412 \text{ or } 3421.
%     \]
%     % Therefore $j\geq n-2$.
%     By Lemma \ref{lemma:fib_structure},
%     $\pi_{n-1}=n$.
%     So by elimination we are left with $i=n-2$.\\
% \end{proof}

% %%%%%%%%%%%%%%%%%%%%%%%%%%%%%%%%%%%%%%%%%%
By Lemma \ref{lemma:fib_structure} and \ref{lemma:fib_structure2}, 
we can partition the set of simple permutations of length $n$ avoiding $P$ by their initial terms.

\begin{definition}
    For $n\geq 4$, define $A_n$, $B_n$ and $C_n$ to be the set of simple $n$-permutations avoiding $P$ 
    begining with $(n-1)\,1\,(n-2)$, $(n-1)\,(n-3)$ and $(n-1)\,1\,(n-3)$ respectively.\\
    Denote their cardinalities by $a_n$, $b_n$ and $c_n$.
\end{definition}

\begin{remark}
    The above definition holds for all $n\geq 4$ even if $A_n$, $B_n$ or $C_n$ are empty. 
    Table \ref{table:fib-summary} summarizes the types of permutations in these sets due to Lemmas \ref{lemma:fib_structure} and \ref{lemma:fib_structure2}.
    Table \ref{table:fib_simples-sizes} gives sample values of $a_n$, $b_n$ and $c_n$ for $4\leq n\leq 8$. 
\end{remark}

\begin{table}[!htbp]
    \centering
    \begin{tabular}{|c|cc|}
        \hline
        Set & \multicolumn{2}{c|}{Permutations in the set are of the form} \\\hline
        $A_n$ & $(n-1)\,1\,(n-2)\,\cdots\, n\, k$ & where $k\in [2,n-3]$ \\    
        $B_n$ & 3142 or $(n-1)\,(n-3)\,\cdots\, (n-2)\, n\, k$ & where $k\in [2,n-4]$ \\
        $C_n$ & $(n-1)\,1\,(n-3)\,\cdots \,(n-2)\, n\, k$ & where $k\in [2,n-4]$ \\\hline
    \end{tabular}
    \label{table:fib-summary}
    \caption{Summary of the types of permutations in $Av_n^S(2413,3412,3421)$ for $n\geq 4$.
    The ellipses represent (possibly empty) factors of the permutations.}
\end{table}


\section{Recursive functions}

We will show that for all $n\geq 6$,
we can obtain simple permutations of length $n$ that avoid $P$
by adding points to smaller simple permutations that avoid $P$.

\begin{definition}
    For $n\geq 6$, let $S_n$ denote the set of permutations on $[n]$, and let 
    \begin{align*}
        f_A:Av_{n-2}^S(P) 
        \rightarrow S_{n}, 
        \quad 
        f_B:Av_{n-2}^S(P) 
        \rightarrow S_{n}, 
        \quad\text{and}\quad
        f_C:B_{n-1} 
        \rightarrow S_{n},
    \end{align*}
    be functions defined as follows:
    \begin{align*}
        f_A(\pi_1 \pi_2\cdots \pi_{n-2}) 
        &:=(n-1)\,1\,(\pi_1+1)\,(\pi_2+1)\, \cdots\,(\pi_{n-3}+2)\,(\pi_{n-2}+1),\\
        f_B(\pi_1 \pi_2\cdots \pi_{n-2}) 
        &:=(n-1)\,\pi_1\,\pi_2\,\cdots\,\pi_{n-3}\, n\,\pi_{n-2},\quad\text{and}\\
        f_C(\pi_1 \pi_2\cdots \pi_{n-1})
        &:=(\pi_1+1)\,1\,(\pi_2+1)\,\cdots\,(\pi_{n-1}+1).
    \end{align*}
\end{definition}

% Recall that for any simple $n$-permutation $\pi$ that avoids $P$ where $n\geq 4$, 
% we have $\pi_1=n-1$ and $\pi_{n-1}=n$ by Lemma \ref{lemma:fib_structure}.
% Also, $B_{n}$ consists of permutations that begin with $(n-1)(n-3)$. 

\begin{definition}
    For $n\geq 6$, let 
    \begin{align*}
        g_A: A_n\rightarrow S_{n-2},
        \quad 
        g_B: B_n\rightarrow S_{n-2}, 
        \quad\text{and}\quad
        g_C: C_n\rightarrow S_{n-1},
    \end{align*}
    be functions defined as 
    \begin{align*}
        g_A(\sigma)&:=(\sigma_3-1)\,(\sigma_4-1)\,\cdots\,(\sigma_{n-2}-1)\,(\sigma_{n-1}-2)\,(\sigma_n-1),\\
        g_B(\sigma)&:=\sigma_2\,\sigma_3\,\cdots\,\sigma_{n-2}\,\sigma_n,\\
        g_C(\sigma)&:=(\sigma_1-1)\, (\sigma_3-1)\, (\sigma_4-1)\, \cdots\, (\sigma_n-1).
    \end{align*}
    where $\sigma=\sigma_1\sigma_2\cdots \sigma_n\in Av_n^S(P)$.
    Note that $g_A(\sigma) = \text{red}(\sigma_{[3,n]})$,
    $g_B(\sigma) = \text{red}(\sigma_{[2,n-1]}\, \sigma_n)$
    and $g_C(\sigma) = \text{red}(\sigma_1\, \sigma_{[3,n]})$.
\end{definition}

% \begin{definition}
%     For $n\geq 6$, define the functions
%     \begin{align*}
%         f_A&:Av_{n-2}^S(P) \rightarrow Av_{n}^S(P),\qquad
%         &&f_A(\pi_1 \pi_2\cdots \pi_{n-2}) 
%         :=(n-1)\,1\,f_A'(\pi_1)\,f_A'(\pi_2)\, \cdots\,f_A'(\pi_{n-2}),\\
%         f_B&:Av_{n-2}^S(P) \rightarrow Av_{n}^S(P),\qquad
%         &&f_B(\pi_1 \pi_2\cdots \pi_{n-2}) 
%         :=(n-1)\,\pi_1\,\pi_2\,\cdots\,\pi_{n-3}\, n\,\pi_{n-2},\\
%         f_C&:B_{n-1} \rightarrow Av_{n}^S(P), \qquad
%         &&f_C(\pi_1 \pi_2\cdots \pi_{n-1})
%         :=f_C'(\pi_1)\,1\,f_C'(\pi_2)\,\cdots\,f_C'(\pi_{n-1}),\\
%     \end{align*}
%     where $f_A',f_C'$ are given by
%     \begin{align*}
%         f_A'&:[n-2]\rightarrow [n],\qquad
%         f_A'(y):=\begin{cases}
%             n&\text{ if }y=n-2,\\
%             y+1&\text{ otherwise},
%         \end{cases}\\
%         f_C'&:[n-1]\rightarrow [n], \qquad 
%         f_C'(y):=\quad y+1.\\
%     \end{align*}
% \end{definition}

% \begin{definition}
%     Let $\sigma=\sigma_1\sigma_2\cdots \sigma_n\in Av_n^S(P)$, and 
%     where
%     \begin{align*}
%         g_A&: A_n\rightarrow Av_{n-2}^S(P),
%         &&g_A(\sigma):=g_A'(\sigma_3) g_A'(\sigma_4) \cdots g_A'(\sigma_n),\\
%         g_B&: B_n\rightarrow Av_{n-2}^S(P),
%         &&g_B(\sigma):=\sigma_2 \sigma_3 \cdots \sigma_{n-2} \sigma_n,\\
%         g_C&: C_n\rightarrow B_{n-1},
%         &&g_C(\sigma):=g_C'(\sigma_1) g_C'(\sigma_3) g_C'(\sigma_4) \cdots g_C'(\sigma_n),
%     \end{align*}
%     where $g_A',g_C'$ are given by
%     \begin{align*}
%         g_A'&:[n]\setminus \{1,n-1\}\rightarrow [n-2],\qquad
%         &&g_A'(y):=\begin{cases}
%             n-2&\text{ if } y=n,\\
%             y-1&\text{ otherwise},
%         \end{cases}\\
%         g_C'&:[n]\setminus\{1\}\rightarrow [n-1], \qquad 
%         &&g_C'(y):=\quad y-1.\\
%     \end{align*}
% \end{definition}

\begin{lemma}\label{fib:f-A}
    If $\pi\in Av_{n-2}^S(P)$ and $n\geq 6$,
    then $f_A(\pi)$ is simple and avoids $P$.
\end{lemma}

\begin{proof}
    Recall that $f_A:Av_{n-2}^S(P) \rightarrow Av_{n}^S(P)$ and
    \[f_A(\pi_1 \,\pi_2\,\cdots \,\pi_{n-2}) 
    :=(n-1)\,1\,(\pi_1+1)\,(\pi_2+1)\,\cdots\,(\pi_{n-3}+2)\,(\pi_{n-2}+1).\]
    % 
    Suppose that $f_A(\pi)_{[i,j]}$ is an interval for some $i,\, j\in [n]$.
    The following cases show that all intervals in $f_A(\pi)$ are of length 0, 1 or $n$
    thereby proving that $f_A(\pi)$ is simple:
    
    \begin{itemize}
        \item Case 1: $i=1$. % $i=1\leq j\leq n$.
        Recall that $f_A(\pi)_1=n-1$ and $f_A(\pi)_2=1$,
        so if $j\geq 2$,
        then $f_A(\pi)_{[i,j]}$ must contain all the numbers in $[n-1]$
        and so must have length at least $n-1$. 
        Observe that $f_A(\pi)_{n-1}=\pi_{n-3}+2=n$.
        So $f_A(\pi)_{[1,j]}$ must contain $[n]$, i.e. $j=n$. 
        
        \item Case 2: $i=2$.
        Observe that $f_A(\pi)_2=1$,
        so $f_A(\pi)_{[2,j]}$ is the interval $[m]$ for some $m\in [n]$.
        If $j>2$, then $f_A(\pi)_{[3,j]}$ is the interval $[2,m]$.
        Recall that $f_A(\pi)_{[3,j]}=(\pi_{1}-1)\,(\pi_{2}-1)\,\cdots \,(\pi_{j-2}-1)$,
        so $\pi_{[1,j-2]}$ is an interval as well,
        and is a trivial interval only if $j\leq 3$.
        We know that $f_A(\pi)_{[2,3]}=1\, (n-3)$ is not an interval,
        so $j\leq 2$.

        \item Case 3: $i\geq 3$.
        If $j<n-1$ then $f_A(\pi)_{[i,j]}=(\pi_{i-2}-1)\,(\pi_{i-1}-1)\,\cdots \,(\pi_{j-2}-1)$.
        So $f_A(\pi)_{[i,j]}$ is an interval only if 
        $\pi_{[i-2,j-2]}$ is an interval,
        which is true only if $j\leq i$ since $j-2<n-3$.
        Note that $f_A(\pi)_{n-1}=\pi_{n-3}+2=n$.
        So if $j\geq n-1$ then $f_A(\pi)_{[i,j]}$ contains the point $n$.
        This means that $f_A(\pi)_{[i,j]}$ can only be a nontrivial interval if it contains $\pi_1=n-1$ as well,
        which is impossible since $i\geq 3$.
        So $f_A(\pi)_{[i,j]}$ is an interval only if $j\leq i$.
    \end{itemize}

    Finally, we check that $f_A(\pi)$ avoids $P$. 
    Suppose we have $1\leq i<j<k<\ell\leq n$ such that  
    $f_A(\pi)_i\, f_A(\pi)_j\, f_A(\pi)_k\, f_A(\pi)_\ell$ reduces to a pattern in $P$. 
    Since $\pi$ avoids $P$, such a subsequence 
    must contain the numbers $n-1$ or $1$. 
    However, the only term larger than $n-1$ is 
    $n=\pi_{n-3}+2=f_A(\pi)_{n-1}$
    by Lemma \ref{lemma:fib_structure}, 
    while all patterns in $P$ have the largest term 4 as the third last number in the pattern,
    so it cannot contain the number $n-1$.
    On the other hand, 
    1 is the second term of $f_A(\pi)$,
    but is the third or fourth term in the patterns in $P$.
    So it cannot contain $1$ either.
    So $f_A(\pi)$ avoids $P$.
\end{proof}
    
 
\bigskip
\begin{lemma}\label{fib:f-B}
    If $\pi \in Av_{n-2}^S(P)$ and $n\geq 6$,
    then $f_B(\pi)$ is simple and avoids $P$.
\end{lemma}

\begin{proof}
    Recall that $f_B:Av_{n-2}^S(P) \rightarrow Av_{n}^S(P)$, and 
    \[f_B(\pi_1 \pi_2\cdots \pi_{n-2}) 
    :=(n-1)\,\pi_1\,\pi_2\,\cdots\,\pi_{n-3}\, n\,\pi_{n-2}.\]

    Suppose that $f_B(\pi)_{[i,j]}$ is an interval for some $i,\, j\in [n]$.
    The following cases show that all intervals in $f_B(\pi)$ are of length 0, 1 or $n$
    thereby proving that $f_B(\pi)$ is simple:

    \begin{itemize}
        \item Case 1: $2\leq i,\, j\leq n-2$.
        Observe that $f_B(\pi)_{[i,j]}=\pi_{[i-1,j-1]}$
        is an interval only if $i\geq j$
        since $\pi$ is simple.
        So $i\geq j$.
        
        \item Case 2: $i=1\leq j\leq n-2$.
        Observe that $f_B(\pi)_{\ell}<f_B(\pi)_1=n-1$ for all $\ell\in [2,j]$.
        So the factor $f_B(\pi)_{[2,j]}=\pi_{[1,j-1]}$ must also be interval.
        Since $\pi$ is simple, and $j-1< n-2$,
        the factor $\pi_{[1,j-1]}$ is an interval only if $j\leq 2$.
        We know that $f_B(\pi)_{[1,2]}=(n-1)\pi_1=(n-1)\,(n-3)$ is not an interval, 
        so we must have $j=1$.

        \item Case 3: $n-1\leq j$.
        Observe that $f_B(\pi)_{n-1}=n$.
        If $i<j$ then $f_B(\pi)_{[i,j]}$ must also contain $n-1=f_B(\pi)_1$, so $i=1$.
        So it is a factor of length at least $n-1$. 
        But $f_B(\pi)_{[1,n-1]}$ is not an interval since it omits $f_B(\pi)_{n}=\pi_{n-2}$
        which is in $[2,n-2]$.
        So implies that $i=1$ and $j=n$.
        Otherwise $i\geq j$.
    \end{itemize}
        
    Finally, we prove that $f_B(\pi)$ avoids $P$.
    Suppose we have $1\leq i<j<k<\ell\leq n$ such that  
    $f_B(\pi)_i\, f_B(\pi)_j\, f_B(\pi)_k\, f_B(\pi)_\ell$ reduces to a pattern in $P$. 
    Then such a subsequence must contain the number $n-1=f_B(\pi)_1$ or $n=f_B(\pi)_{n-1}$ 
    since we know that $\pi$ avoids $P$.
    Clearly it cannot contain the number $n$ since 
    it is the second last term in the permutation,
    but 4 never occurs as the last or second last term in patterns in 2413, 3412 or 3421.
    Then we must have $i=1$.
    But no patterns in $P$ start with 4, 
    so $f_B(\pi)$ indeed avoids $P$.
\end{proof}

\bigskip
\begin{lemma}\label{fib:f-C}
    If $\pi \in B_{n-1}$ and $n\geq 6$,
    % is an $(n-1)$-permutation simple with $\pi_2=n-4$ and $n\geq 7$. 
    then $f_C(\pi)$ is simple and avoids $P$.
\end{lemma}
\begin{proof}
    Recall that $f_C:B_{n-1} \rightarrow Av_{n}^S(P)$ and
    \[f_C(\pi_1\, \pi_2\, \cdots\, \pi_{n-1})
    :=(\pi_1+1)\,1\,(\pi_2+1)\,\cdots\,(\pi_{n-1}+1).\]

    It is easy to see that $f_C(\pi)$ avoids $P$.
    The only way that $f_C(\pi)$ could contain $P$ due to the addition of the point 1
    is if there are at least 2 terms preceding it.
    However, it is inserted in the second position, 
    so $f_C(\pi)$ does not contain $P$.\\
    
    Suppose that $f_C(\pi)_{[i,j]}$ is an interval for some $i,\, j\in [n]$.
    The following cases show that all intervals in $f_B(\pi)$ are of length 0, 1 or $n$
    thereby proving that $f_C(\pi)$ is simple:
    
    \begin{itemize}
        \item Case 1: $i=1$. 
        Suppose $j\geq 2$.
        Observe that $f_C(\pi)_1=\pi_1+1=n-1$ and $f_C(\pi)_2=1$.
        Then the interval $f_C(\pi)_{[i,j]}$ 
        must contain all the numbers in $[n-1]$.
        This includes $f_C(\pi)_n = \pi_{n-1} + 1$ which is in $[3,n-4]$ for $n\geq 6$.
        This is because $\pi_{n-1}$ cannot be $n-4$, $n-3$ or $n-2$  
        by Lemmas \ref{lemma:fib_structure} and \ref{lemma:fib_structure2},
        and is not $1$ or $n-1$ since $\pi$ is simple.
        So $j=n$.

        \item Case 2: $i=2$.
        Observe that $f_C(\pi)_2=1$ and $f_C(\pi)_3=\pi_2+1=n-3$. 
        If $j> 2$, then $f_C(\pi)_{[i,j]}$ is an interval only if it contains $[n-3]$.
        So the factor must have length at least $n-3$,
        meaning that $j\geq n-2$.
        Observe that $f_C(\pi)_{n-2}=\pi_{n-2}+1=n-1$ by Lemma \ref{lemma:fib_structure2},
        so $f_C(\pi)_{[i,j]}$ must contain $[n-1]$.
        However, this is not the case since it omits $f_C(\pi)_1=\pi_1+1$ which is not $n$.
        So $j\leq 2$.
        
        \item Case 3: $i\geq 3$.
        Observe that $f_C(\pi)_{[i,j]} =(\pi_{i-1}+1)\,(\pi_{i}+1)\,\cdots\,(\pi_{j-1}+1)$.
        Since $\pi$ is simple, 
        the factor $\pi_{[i-1,j-1]}$ 
        is an interval only if $i \geq j$.
        So $i \geq j$.
    \end{itemize}
\end{proof}    

\begin{lemma}\label{fib:g-A}
    If $\sigma=(n-1)\,1\,(n-2)\,\cdots\, n \, k\in A_{n}$ and $n\geq 6$,
    then $g_A(\sigma)$ is simple and avoids $P$.
\end{lemma}

\begin{proof}
    The mapping $g_A$ removes the first two entries and then reduces the result.
    Since $g_A(\sigma)$ is a reduced subsequence of $\sigma$, 
    it avoids $P$.\\
    
    Suppose that $g_A(\sigma)$ is not simple.
    Then there exists $[a,b] \subsetneq [3,n]$ and $[c,d] \subseteq [2,n]$ 
    with $d \neq n-1$  and $b \geq a+2$ such that
    $\{\sigma_i: i\in [a,b]\}= [c,d]^0$ where $[c,d]^0 := [c,d]\setminus \{n-1\}$.\\
    
    If $d = n$ then $n-2 \in [c,d]^0$. Hence $a=3$ and $2 \in \{\sigma_i: i\in [a,b]\} = [c,d]^0$ 
    which implies that $k = \pi_n \in [c,d]^0$ as well. 
    But we have excluded the case $[a,b]=[3,n]$ 
    since it is a trivial interval for $g_A(\sigma)$. 
    This shows that $d \leq n-2$.
    But then $[c,d]^0=[c,d]$ and $\{\sigma_i: i\in [a,b]\}=[c,d]$ is an interval for $\sigma$.

        % Suppose $g_A(\sigma)_{[i,j]}$ is an interval
        % for some $i,\, j\in [n-2]$.
        % Then we have the following cases:
        
        % \begin{itemize}
        %     \item Case 1: $j\geq n-3$.
        %     We know that $g_A(\sigma)_{[n-3,n-2]}=(\sigma_{n-1}-2)\, (\sigma_{n}-1)$ 
        %     is not an interval.
        %     So suppose $i<j=n-3$.
        %     By Lemma \ref{lemma:fib_structure}, 
        %     $g_A(\sigma)_{n-3}=\sigma_{n-1}-2=n-2$,
        %     while $g_A(\sigma)_1=(n-2)-1$ by definition of $A_n$.
        %     So we must have $i=1$.
        %     By Corollary \ref{cor:fib-pin},
        %     $g_A(\sigma)_{n-2}=\sigma_n-1\in [2,n-4]$,
        %     so $g_A(\sigma)_{[1,n-3]}$ is not an interval.
        %     So we are left with our final option that is indeed valid,
        %     i.e. $i=1$ and $j=n-2$.
    
        %     \item Case 2: $j\leq n-4$.
        %     Observe that $g_A(\sigma)_{[i,j]}=(\sigma_{i+2} -1)(\sigma_{i+3} -1)\cdots (\sigma_{j+2} -1)$.
        %     So if $g_A(\sigma)_{[i,j]}$ is an interval,
        %     %  $[a,b]$ for some $a,b\in [n-2]$,
        %     then so is $\sigma_{[i+2,j+2]}$.
        %     %  is the interval $[a-1,b-1]$.
        %     Since $\sigma$ is simple and $i+2\geq 3$, 
        %     $\sigma_{[i+2,j+2]}$ is an interval only if $i=j$.
        %     So $g_A(\sigma)_{[i,j]}$ is an interval only if $i=j$.
        % \end{itemize}
            
        % \noindent Since the only intervals in $g_A(\sigma)$ are of length 0, 1 or $n-2$,
        % $g_A(\sigma)$ is simple.
\end{proof}

\begin{lemma}\label{fib:g-B}
    If $\sigma\in B_{n}$ and $n\geq 6$,
    then $g_B(\sigma)$ is simple and avoids $P$.
\end{lemma}
    
\begin{proof}
    The mapping $g_B$ removes the first entry and the second last entry of $\sigma$.
    Since $g_B(\sigma)$ is a reduced subsequence of the permutation $\sigma$, 
    it avoids $P$.\\
    
    Suppose that $g_B(\pi)$ is not simple.  
    Then there exists $[a,b] \subsetneq [2,n]$ and $[c,d] \subsetneq [1,n-2]$ 
    with $b \neq n-1$ and $c<d$ such that
    $\{\pi_{i}:i\in [a,b]^0\} = [c,d]$ 
    where $[a,b]^0 := [a,b] \setminus \{n-1\}$.  
    %    
    If $b = n$ then $n-2 \in [a,b]^0$.  
    Therefore $n-2 =\pi(n-2) \in [c,d]$ 
    and that implies that $(n-3) \in [c,d]$. 
    Hence $a=2$ which means $[a,b]=[2,n]$.
    This contradiction implies that $b \leq n-2$.
    But then $[a,b]^0=[a,b]$ and $\pi_{[a,b]}=[c,d]$ is an interval for $\pi$.

    % Observe that in the map $\sigma\mapsto g_B(\sigma)$,
    % only the numbers $n-1$ and $n$ are removed,
    % while the others remain unchanged in both position and value.
    % Suppose the factor $g_B(\sigma)_{[i,j]}$ 
    % is an interval $[a,b]$ for some $i,j,a,b\in [n-2]$.

    % By Corollary \ref{cor:fib-pin},
    % $g_B(\sigma)_{n-3}=\sigma_{n-2}=n-2$.

    % \begin{itemize}
    %     \item Case 1: $2\leq i\leq j\leq n-3$.
    %     Observe that $g_B(\sigma)_{[i,j]}=\sigma_{[i+1,j+1]}$.
    %     Since $\sigma$ is simple and $j+1<n$,
    %     the factor 
    %     $\sigma_{[i+1,j+1]}$ 
    %     is an interval only if $i=j$.

    %     \item Case 2: $i=1\leq j\leq n-3$.
    %     From Corollary \ref{cor:fib-pin},
    %     we know that $g_B(\sigma)_{n-2}=\sigma_n\in [2,n-3]$,
    %     so $g_B(\sigma)_{[1,n-3]}$ is not an interval. 
    %     Therefore, $g_B(\sigma)_{[i,j]}$ 
    %     is an interval only if $j=1$.

    %     \item Case 2: $j= n-2$. 
    %     Suppose $i<j$.
    %     \todo{}
    % \end{itemize}

    % \noindent So all intervals in $g_B(\sigma)$ are of length $0,1$ or $n-2$,
    % proving that $g_B(\sigma)$ is simple.
\end{proof}
    
\begin{lemma}\label{fib:g-C}
    If $\sigma\in C_{n}$ and $n\geq 6$,
    then $g_C(\sigma)$ is simple and avoids $P$.
\end{lemma}

\begin{proof}
    % Recall that $g_C: C_n\rightarrow B_{n-1}$ and 
    % \[g_C(\sigma):=(\sigma_1-1) (\sigma_3-1) (\sigma_4-1) \cdots (\sigma_n-1).\]
    The mapping $g_C$ removes the second entry (which is 1) and then reduces the result.
    It is not hard to see that $g_C(\sigma)$ is a reduced subsequence of the permutation $\sigma$, 
    so it must avoid $P$.\\

    Suppose that $g_C(\pi)$ is not simple.  
    Then there exists $[a,b] \subsetneq [1,n]$ and $[c,d] \subsetneq [2,n]$ 
    with $a \neq 2$ and $d \geq c+2$ such that
    $\{\pi_i: i\in [a,b]^0\} = [c,d]$ 
    where $[a,b]^0 := [a,b] \setminus \{2\}$.  

    If $a = 1$ then $n-1 \in [c,d]$.  
    Therefore $n-2$ or $n$ lies in $[c,d]$.  
    Thus $b \geq n-3$ and this implies 
    $2 \in \pi([a,b]^0) = [c,d]$.  
    Therefore $k = \pi_n \in [c,d]$ and thus $b = n$, 
    but we have excluded the case where $[a,b]=[1,n]$.  
    This contradiction implies that $a \geq 3$.
    But then $[a,b]^0=[a,b]$ and $\pi_{[a,b]}=[c,d]$ is a nontrivial interval for $\pi$.

    % Now we show that $g_C(\sigma)$ is simple.
    % Suppose $g_C(\sigma)_{[i,j]}$ is an interval
    % for some $i,\, j\in [n-1]$.
    % Then we have the following cases:
    
    % \begin{itemize}
    %     \item Case 1: $i=1$.
    %     Recall that $\sigma_1=2$ by the definition of $C_n$.
    %     So if $g_C(\sigma)_{[1,j]}$ is an interval $[a,b]$ for some $a\leq b$,
    %     then we must have that 
    %     $\sigma_{[1,j+1]}$ is the interval $[1,b]$.
    %     Since $\sigma$ is simple, this occurs 
    %     only if $ j=n$.
    %     So $g_C(\sigma)_{[1,j]}$ is an interval only if 
    %     $j=n$ or 1.

    %     \item Case 2: $i\geq 2$.
    %     Observe that $g_C(\sigma)_{[i,j]}=(\sigma_{i+1}-1)(\sigma_{i+2}-1)\cdots (\sigma_{j+1}-1)$.
    %     Since $\sigma$ is simple and $i+1\geq 3$, 
    %     $\sigma_{[i+1,j+1]}$ is an interval only if $i=j$.
    %     We know that $\sigma_1=n-1$, so $g_C(\sigma)_1=n-2$
    %     and $g_C(\sigma)_{[2,n-1]}$ is not an interval.
    %     So we must have $i\geq j$.  
    % \end{itemize}
\end{proof}

\begin{table}[!htbp]
    \begin{center}
        \begin{tabular}{ | c | c | c | c | c | c |}
            \hline
            $n$ & \thead{\makecell{$A_n:=$ permutations\\that start with \\$(n-1)\,1\,(n-2)$}} & \thead{\makecell{$B_n:=$ permutations\\that start with \\$(n-1)\,(n-3)$}} & \thead{\makecell{$C_n:=$ permutations\\that start with \\$(n-1)\,1\,(n-3)$}} \\\hline
            4   &   -       &   3142    &   -       \\
            5   &   41352   &   -       &   -       \\
            6   &   514263  &   531462  &   -       \\
            7   &   6152473 &   6413572 & 6142573   \\
            8   & \makecell{71625384,\\71642583}    & \makecell{75142683,\\75314682}  &   71524683    \\
            \hline
        \end{tabular}
        \caption{Simple $n$-permutations avoiding $2413,\,3412,\,3421$ for $4\leq n\leq 8$}
        \label{table:fib_simples-perms}
    \end{center}
\end{table}

\begin{table}[!htbp]
    \begin{center}
        \begin{tabular}{ | c | c | c | c | c | c |}
            \hline
            $n$ & $Av_n^S(P)$ & $\abs{Av_n^S(P)}$ & $a_n$ & $b_n$ & $c_n$ \\\hline
            % $n$ & $Av_n^S(P)$ & $\abs{Av_n^S(P)}$ & \thead{\makecell{$a_n:=$ number\\of permutations\\that start with \\$(n-1)\,1\,(n-2)$}} & \thead{\makecell{$b_n:=$ number\\of permutations\\that start with \\$(n-1)\,(n-3)$}} & \thead{\makecell{$c_n:=$ number\\of permutations\\that start with \\$(n-1)\,1\,(n-3)$}} \\\hline
            4   & 3142                                     & 1        & 0               & 1           & 0\\
            5   & 41352                                    & 1        & 1               & 0               & 0\\
            6   & \makecell{514263,\\531462}               & 2        & 1               & 1               & 0\\
            7   & \makecell{6152473, \\6142573, \\6413572} & 3        & 1               & 1               & 1\\
            8   & \makecell{71625384,\\71642583,\\71524683,\\75142683,\\75314682} & 5        & 2               & 2               & 1\\
            $k\geq 6$ & \makecell{$Av_k^S(P)$} & $F(k-3)$ & $F(k-5)$ & $F(k-5)$ & $F(k-6)$\\
            \hline
        \end{tabular}
        \caption{The number of simples avoiding $2413,\,3412,\,3421$, where $4\leq n\leq 8$}
        \label{table:fib_simples-sizes}
    \end{center}
\end{table}

\section{Proof of Theorem \ref{thm:fib}}

It is well known that there are no simple permutations of length $3$, 
so it is obvious that $Av_3^S(P)=0=F(0)$.
For $n=4,\,5,\,6,\,7$, the values in Table \ref{table:fib_simples-sizes} are easy to verify.\\

We proceed to prove the enumeration for $n\geq 6$ via induction.
Recall that every simple $n$-permutation 
avoiding $P$ begins with $n-1$ 
by Lemma \ref{lemma:fib_structure}.
So for all $n\geq 6$, if $\pi\in Av_{n-2}^S(P)$ then we have 
\begin{align*}
    f_A(\pi_1 \pi_2\cdots \pi_{n-2})_{[1,3]}\quad
    &=\quad(n-1)\,1\,(\pi_1+1)\\
    &=\quad (n-1)\,1\,(n-2)
    \quad \text{and}\\
    f_B(\pi_1 \pi_2\cdots \pi_{n-2})_{[1,2]}\quad
    &=\quad (n-1) \, \pi_1\\
    &=\quad (n-1) \, (n-3),
\end{align*}

\noindent and for all $n\geq 6$ and $\pi\in B_{n-1}$ we have 
\begin{align*}
    f_C(\pi_1 \pi_2\cdots \pi_{n-1})_{[1,3]}\quad 
    &=\quad (n-1) \, 1 \, (\pi_2 +1)\\
    &=\quad (n-1) \, 1 \, (n-3).
\end{align*}

\noindent Therefore, from Lemmas \ref{fib:f-A}, \ref{fib:f-B} and \ref{fib:f-C},
\begin{align*}
    \pi\in Av_{n-2}^S(P) &\implies f_A(\pi)\in A_n
    \quad \text{and} \quad f_B(\pi)\in B_n,\\
    \text{and}\quad \pi\in B_{n-1} &\implies f_C(\pi)\in C_n,
\end{align*}

\noindent which means that $f_A(Av_{n-2}^S(P)) \subseteq A_n$, 
$f_B(Av_{n-2}^S(P)) \subseteq B_n$ and $f_C(B_{n-1}^S(P)) \subseteq C_n$. 
% \[
% f_A(Av_{n-2}^S(P)) \subseteq A_n,\quad 
% f_B(Av_{n-2}^S(P)) \subseteq B_n,\quad 
% f_C(B_{n-1}^S(P)) \subseteq C_n.
% \]

\noindent On the other hand,
Lemmas \ref{fib:g-A},\ref{fib:g-B} and \ref{fib:g-C} clearly show that 
\begin{align*}
    \sigma\in A_{n} &\implies g_A(\sigma)\in Av_{n-2}^S(P)
    \quad \text{and} \quad \\
    \sigma\in B_{n} &\implies g_B(\sigma)\in Av_{n-2}^S(P)
    \quad \text{and} \quad \\
    \sigma\in C_{n} &\implies g_C(\sigma)\in B_{n-1}
    \quad \text{and} \quad 
\end{align*}

\noindent which means that
$g_A(A_{n}) \subseteq Av_n^S(P)$,
$g_B(B_{n}) \subseteq Av_n^S(P)$ and
$g_C(C_{n}) \subseteq B_{n-1}$.
% \begin{align*}
%     g_A(A_{n}) \subseteq Av_n^S(P),\\
%     g_B(B_{n}) \subseteq Av_n^S(P),\\
%     g_C(C_{n}) \subseteq B_{n-1}.
% \end{align*}
% 
\noindent Therefore, 
$g_A$, $g_B$ and $g_C$ is in fact the inverse of 
$f_A$, $f_B$ and $f_C$ respectively.
% 
Moreover, we claim that for all $n\geq 6$,
\[a_n = \abs{Av_{n-2}^S(P)},\quad
b_n = \abs{Av_{n-2}^S(P)},
\quad \text{and}\quad 
c_n = b_{n-1}.\]

\noindent Suppose that for some $k\geq 6$,
\[a_\ell=F(\ell-5), \quad 
b_\ell=F(\ell-5), 
\quad \text{and}\quad 
c_{\ell}=F(\ell-6) \]
for all $\ell\in\{4,5,\dots,k-1\}$.
% 
Then from the statements above,
\begin{align*}
    a_{k}
    &=\abs{Av_{k-2}^S(P)}\\
    &=a_{k-2}+b_{k-2}+c_{k-2}\\
    &=F(k-7)+F(k-7)+F(k-8)\\
    &=F(k-7)+F(k-6)\\
    &=F(k-5), \\
    \text{so} \quad b_{k}
    =\abs{Av_{k-2}^S(P)}
    &=F(k-5)
    \quad \text{and}\quad
    c_k
    =b_{k-1}
    =F(k-6).
\end{align*}

\noindent So the claim is true by induction.
Therefore, for all $n\geq 6$,
\begin{align*}
    \abs{Av_n^S(P)}
    &=a_{n}+b_{n}+c_{n}\\
    &=F(n-5)+F(n-5)+F(n-6)\\
    &=F(n-5)+F(n-4)\\
    &=F(n-3).
\end{align*}

\noindent This concludes the proof of Theorem \ref{thm:fib}. 
\qed
\end{document}