% !TEX root = /Users/keshiayap/Desktop/Math/3\ Summer\ 2020/masters/final/main/main.tex
\documentclass[../main.tex]{subfiles}
%
\begin{document}
\section{Summary}

In this thesis, we elucidated connections between the avoidance sets of some POPs and other combinatorial objects by constructing explicit bijections between the relevant sets,
as a direct response to five of the 15 open questions posed by Gao and Kitaev \cite{gao-kitaev-2019}.
These bijections were derived primarily by analysing the simple permutations of the avoidance sets and how the rest of the set could be obtained from their inflations.
This was made possible by illustrating the permutation matrices as lattice matrices, which is a novel concept introduced in this paper.
The bijections constructed in this paper are a testament to the fundamental role that simple permutations play in the study of pattern-avoiding permutations.
It also demonstrates the intricate connections that avoidance sets of POPs have with many other combinatorial objects,
and provides a way to relate seemingly disparate combinatorial objects through their connections to the family of POPs.
We also enumerated the number of simple $n$-permutations avoiding the patterns 2413, 3412 and 3421 for all $n$,
giving a concrete example of an avoidance set with a finite basis and infinitely many simple permutations.\\

\section{Further Work}

The remaining ten questions posed by Gao and Kitaev \cite{gao-kitaev-2019} remain open. 
Given the bijections we have constructed, it would be interesting to know whether they can be generalized further, 
by studying generalizations of the POPs or of the combinatorial objects.
The following questions are natural extensions of the problem that was discussed in Chapter \ref{chap:levels}:
\begin{enumerate}[1.]
    \item Are there any combinatorial objects that have a natural bijective relationship with the avoidance set of $P_k$ for any $k\geq 5$?
    \item Is there a POP whose avoidance set is in bijection with the levels in compositions of ones, twos and threes?
    \item Are there any combinatorial objects that have a natural bijective relationship with the avoidance set 
    of the POP with $k$ elements labelled $1,\, 2,\, \dots,\, k$ where $1>3>5$,
    or, more generally, with  $1>3>\cdots > 2i+1$ for some $i\geq 2$? 
\end{enumerate}

The enumeration of $Av_n(R_k)$ for $k\geq 6$, $n\geq 1$, where $R_k$ is defined in Chapter \ref{chap:av10}, is an open question.
It could also be interesting to enumerate $\mathfrak{S}_{k,n}$,
which we define as the set of permutations whose partial sums of signed displacements do not exceed $k$, for all $k\geq 3$,
and check if there exist any $k$ and $\ell$ such that 
$\abs{\mathfrak{S}_{k,n}}=\abs{Av_n(R_\ell)}$ for all $n\geq 1$.
Finally, Gao and Kitaev \cite{gao-kitaev-2019} observed that sequence $\abs{Av_{n-1}(R_4)}_{n\geq 2}$ 
corresponds to sequence \href{https://oeis.org/A232164}{A232164} as well.
The latter sequence counts the number of 
Weyl group elements, not containing an $s_r$ factor, which contribute nonzero terms to Kostant's weight multiplicity formula when computing the multiplicity of the zero-weight in the adjoint representation for the Lie algebra of type $C$ and rank $n$.
Using our analysis on the set $Av(R_4)$, 
one may be able to construct a natural bijection between these two sets more easily.\\

During our study of the simple permutations that avoid the patterns 2413, 3412 and 3421, 
we discovered using the PermLab software that the addition of the pattern 2431 to the basis does not change the set of simple permutations for small $n$.
It can be proved that the simple permutations constructed by the recursive functions 
to build the set $Av_n^S(2413,3412,3421)$ indeed avoid 2431.
This observation leads us to an interesting question: 
Which avoidance sets have the same set of simples?

% One key observation is that for all $k\geq 2$, there are finitely many simples in $Av_n(R_k)$ (to prove). 
% If we can show that all but finitely-many permutations are sum-decomposable, 
% we simply need to show that for all $n\leq N$ for some $N\geq 1$, 
% the number of permutations in $\mathfrak{S}_{k,n}$
% is equal to the number of sum-indecomposable permutations in $Av_n(R_k)$.
\end{document}