\documentclass[xcolor=table]{beamer}

\usepackage[sorting=ynt,block=ragged,backend=bibtex8]{biblatex}
\addbibresource{references.bib}

% Colorful tables
\usepackage{multirow} 

\usepackage{csquotes}
% For posets
\usepackage{tikzit}
\input{node.tikzstyles}

% For multiline tables
\usepackage{fourier} 
\usepackage{array}
\usepackage{makecell}
\renewcommand\theadalign{bc}
\renewcommand\theadfont{\bfseries}
\renewcommand\theadgape{\Gape[4pt]}
\renewcommand\cellgape{\Gape[4pt]}

% Content 
\usepackage{ulem} % Underlining for emphasis
\usepackage{hyperref} % Hyperlinks
\usepackage{mathtools}
\usepackage{amsthm} % For proof env and others
\usepackage{amssymb} % \varnothing
\usepackage{enumerate} % Allows for enumerate tagging shortut

% Math mode strikeout
\usepackage{xcolor,cancel}
\newcommand\hcancel[2][black]{\setbox0=\hbox{$#2$}%
\rlap{\raisebox{.45\ht0}{\textcolor{#1}{\rule{\wd0}{1pt}}}}#2} 

% For declaring large plus signs
\DeclareMathOperator*{\Plus}{\scalerel*{\color{gray} +}{\textstyle\sum}}
\DeclareMathOperator*{\Bullet}{\scalerel*{\bullet}{\textstyle\sum}}

\newcommand{\inv}{{^{-1}}} % Inverse
\newcommand{\id}{\text{Id}}
\newcommand{\ds}{\displaystyle}
\newcommand{\Z}{{\mathbb Z}}
\newcommand{\N}{{\mathbb N}}
\newcommand{\bc}[1]{{\quad \text{(#1)}}} 	% Justification in math env
\newcommand{\abs}[1]{{\left |#1\right |}}  % Abs value / Cardinality
\newcommand{\ontop}[2]{\stackrel{\mathclap{\normalfont\mbox{#2}}}{#1}} % Text above equations
\newcommand{\circled}[1]{\raisebox{.5pt}{\textcircled{\raisebox{-.9pt} {#1}}}} % Custom circled symbols
\newcommand{\udots}{\reflectbox{$\ddots$}}  
\newcommand{\defterm}[1]{\textcolor{orange}{#1}}
% Markers 
% \theoremstyle{plain} %  sets the text in italic and adds extra space above and below
% \newtheorem{theorem}{Theorem}[section]
% \newtheorem{lemma}[theorem]{Lemma}
% \newtheorem{corollary}{Corollary}[theorem]
\newtheorem{prop}{Proposition}[section]

% \theoremstyle{definition} % adds extra space above and below, but sets the text in roman
% \newtheorem{definition}{Definition}[section]
% \newtheorem{notation}{Notation}[section]
% \newtheorem{conjecture}{Conjecture}
% \newcommand{\labelenumi}{\bf \alph{enumi}.} % enumerate uses a., b., c., ...

% \theoremstyle{remark} % set in roman, with no additional space above or below
% \newtheorem{remark}{Remark}
% \newtheorem{observation}{Observation}
% \renewtheorem{example}{Example}[section]

% Stack Exchange - How can I position an image in an arbitrary position in beamer?
\tikzset{
  every overlay node/.style={
    draw=white,fill=white,anchor=north west,
  },
}
% Usage:
% \tikzoverlay at (-1cm,-5cm) {content};
% or
% \tikzoverlay[text width=5cm] at (-1cm,-5cm) {content};
\def\tikzoverlay{%
   \tikz[baseline,overlay]\node[every overlay node]
}%

% There are many different themes available for Beamer. A comprehensive
% list with examples is given here:
% http://deic.uab.es/~iblanes/beamer_gallery/index_by_theme.html
% You can uncomment the themes below if you would like to use a different
% one:
% \usetheme{AnnArbor}
% \usetheme{Antibes}
% \usetheme{Bergen}
% \usetheme{Berkeley}
% \usetheme{Berlin}
% \usetheme{Boadilla} 
% \usetheme{boxes} % Option 2
% \usetheme{CambridgeUS} % Option 3
% \usetheme{Copenhagen}
% \usetheme{Darmstadt}
% \usetheme{default}
% \usetheme{Frankfurt}
\usetheme{Goettingen} % Option 4
% \usetheme{Hannover} % Option 5
% \usetheme{Ilmenau}
% \usetheme{JuanLesPins}
% \usetheme{Luebeck}
% \usetheme{Madrid}
% \usetheme{Malmoe}
% \usetheme{Marburg} % Option 6
% \usetheme{Montpellier} % Option 7
% \usetheme{PaloAlto}
% \usetheme{Pittsburgh}
% \usetheme{Rochester}
% \usetheme{Singapore}
% \usetheme{Szeged}
% \usetheme{Warsaw}

\usepackage[english]{babel}
\usepackage{bbm}
\usepackage{graphicx}
\usepackage{pgfplots}
\setbeamertemplate{footline}[frame number] % Number the slides

\pgfplotsset{compat=1.12}
\usetikzlibrary{patterns}

\title{Bijections from the Avoidance Sets of Partially Ordered Patterns}

\subtitle{Masters Thesis Project}

\author{Keshia Yap}
% \small{Advisors: David Wehlau \& Imed Zaguia}
\institute[Queen's University] % (optional, but mostly needed)
{
{\normalsize Queen's University}\\
}

\date{September 22, 2020}

\subject{Combinatorics}
% This is only inserted into the PDF information catalog. Can be left
% out. 

\pgfdeclareimage[height=2cm]{university-logo}{QueensLogo-colour.jpg}
 \logo{\pgfuseimage{university-logo}}

% Delete this, if you do not want the table of contents to pop up at
% the beginning of each subsection:
\AtBeginSubsection[]
{
  \begin{frame}<beamer>{Outline}
    \tableofcontents[currentsection,currentsubsection]
  \end{frame}
}

\begin{document}

\begin{frame}
  \titlepage
\end{frame}

\begin{frame}{Outline}
  \tableofcontents [pausesections]
  % You might wish to add the option [pausesections]
\end{frame}

% Section and subsections will appear in the presentation overview
% and table of contents.
\section{Introduction}

\subsection{What is a pattern?}
    
\begin{frame}{What is a pattern?}
    \begin{example}
        \begin{figure}
            \only<1>{\tikzfig{./}{5463721}}
            \only<2>{\tikzfig{./}{red5463721}}
            \only<3->{\tikzfig{./}{red(572)}}
            \caption{Graph of the permutation 5463721}
        \end{figure}
        The permutation 5463721 has the subsequence \textcolor{red}{572},
        which \textit{reduces} to \textcolor{red}{$231$}
        Write red$(572)=231$ and
        say that 5463721 \textit{contains} the pattern 231.
    \end{example}
\end{frame}

\begin{frame}{Pattern Reduction
    \tikzoverlay[text width=0.01cm] at (4.15cm,2cm) {
    \begin{figure}
        \scalebox{0.6}{\tikzfig{./}{red5463721}}
    \end{figure} 
    };
    }
    \begin{definition}
        Let $\pi$ be a permutation on $[n]:=\{1,\,2,\,\dots,\,n\}$
        and $1\leq i_1<i_2<\cdots <i_k\leq n$.
        We define the \defterm{reduction} of 
        the subsequence $\pi_{i_1}\,\pi_{i_2}\,\cdots\,\pi_{i_k}$.
        as the $k$-permutation \textit{order-isomorphic} to it,
        and denote it as \defterm{red$(\pi_{i_1}\,\pi_{i_2}\,\cdots\,\pi_{i_k})$}.\\\vspace{1cm}

        That is, $\text{red}(\pi_{i_1}\pi_{i_2}\,\cdots\,\pi_{i_k}) = \sigma_1\sigma_2\cdots \sigma_k$ 
        if and only if \[ \pi_{i_s}<\pi_{i_t} \iff \sigma_{s}<\sigma_{t}.\]
    \end{definition}
\end{frame}

\begin{frame}{Pattern Containment and Avoidance
    \tikzoverlay[text width=0.01cm] at (0.5cm,2cm) {
    \begin{figure}
        \scalebox{0.6}{\tikzfig{./}{red5463721}}
    \end{figure} 
    };
    }
    \begin{definition} % [Containment and avoidance of patterns]
        A \defterm{pattern} is a permutation of length at least 2.
    \end{definition}
    \pause
    \begin{definition}
        We say that a permutation $\pi$ \defterm{contains} a pattern $p$ if and only if 
        there is a subsequence $\pi_{i_1}\pi_{i_2}\,\cdots\,\pi_{i_k}$ of $\pi$ 
        % where red$(\pi_{i_1}\pi_{i_2}\,\cdots\,\pi_{i_k})=p$. 
        where \[\text{red}(\pi_{i_1}\pi_{i_2}\,\cdots\,\pi_{i_k})=p.\]
        That is, $p_j<p_\ell$ if and only if $\pi_{i_j}<\pi_{i_\ell}$.\\
        Otherwise, we say that $\pi$ \defterm{avoids} $p$.
    \end{definition}
    \pause
    \begin{example}
        \begin{itemize}
            \item 5463721 contains 231
            \item 1324 avoids 231
        \end{itemize}
    \end{example}
\end{frame}

\begin{frame}{Pattern Containment and Avoidance
    \tikzoverlay[text width=0.01cm] at (0.5cm,2cm) {
    \begin{figure}
        \scalebox{0.6}{\tikzfig{./}{red5463721}}
    \end{figure} 
    };
    }
    \begin{definition}
        Let $p$ be a pattern. Define
        \begin{itemize}
            \item $Av(p)$ 
            $:=$ \defterm{avoidance set} of $p$\\
            $:=$ the set of all permutations that avoid $p$
            \item $Av_n(p):=Av(p)\cap S_n$
        \end{itemize}\vspace{1cm}
        \pause 
        Let $P$ be a set of patterns. Define
        \begin{itemize}
            \item $Av(P):=\bigcap_{p\in P} Av(p)$
            \item $Av_n(P):=Av(P)\cap S_n$
        \end{itemize}
    \end{definition}
\end{frame}

\subsection{POPs}

\begin{frame}{Partially Ordered Patterns (POPs)}
    Sometimes, the order between certain elements in a pattern may not be important.
    We are left with a \textit{partially ordered pattern}, which we call a POP:
    \begin{figure}
        \only<1>{\tikzfig{./}{231}}
        \only<2>{\tikzfig{./}{red231}}
        \only<3>{\tikzfig{./}{121}}
        \only<5->{\tikzfig{./}{231p}}
        \only<4->{\tikzfig{./}{121-POP}}
        \only<5->{\tikzfig{./}{132}}
        \caption{Graph of the pattern 231 and a POP generalization of 231}
    \end{figure}
    \only<6>{
    \begin{definition} 
        A \defterm{partially ordered pattern} (POP) of size $k$ is a partially ordered set (poset)
        with $k$ elements labelled $1,\,2,\,\dots,\,k$. \\
    \end{definition}}
\end{frame}
    
\begin{frame}{POP Containment/Avoidance}
    \begin{definition}
        An $n$-permutation $\pi$ \defterm{contains} a POP $P$ if and only if 
        $\pi$ has a subsequence $\pi_{i_1}\pi_{i_2}\cdots \pi_{i_k}$,
        % where $1\leq i_1<i_2<\cdots <i_k\leq n$ 
        such that $\pi_{i_j} < \pi_{i_m}$ if $j<m$ in the poset $P$. \\
        Otherwise, we say that $\pi$ \defterm{avoids} $P$.
    \end{definition}

    \begin{example}
        The permutation \textcolor<2>{red}{34}7\textcolor<2>{red}{26}15 contains \tikzfig{./}{example} while 132456 avoids it.
    \end{example}
\end{frame}

\begin{frame}
    POPs generalize classical patterns.
    \begin{example}
        The pattern 3241 represented as a POP is the 4-element chain with its elements labelled 1, 2, 3 and 4, where $4<2<1<3$.
    \end{example}
    \begin{figure}
        \tikzfig{./}{3241}
        \caption{The POP representation of the pattern 3241}
    \end{figure}
    Note that the permutation 4213 is the inverse of 3241.
\end{frame}

% \begin{frame}
% \begin{example}
%     The POP {\tikzfig{./}{example}} is of length 4 with $1>3$ and\\
%     represents the patterns 2314, 2413, 3124, 3421, 3214, 3412, 4213, 4312, 4123, 4321, 4132, 4231.\\
% \end{example}
% \end{frame}

\subsection{Simple Permutations}

\begin{frame}{Intervals and Factors}
    \begin{definition}
        For an $n$-permutation $\pi$ and any $1\leq i\leq j\leq n$,
        the contiguous substring $\pi_i\pi_{i+1}\,\cdots\,\pi_j$ 
        is called a \defterm{factor} of $\pi$.
        Let \[\pi_{[i,j]}:=\pi_i\pi_{i+1}\,\cdots\,\pi_j.\]
        % Denote $\pi_i\pi_{i+1}\,\cdots\,\pi_j$ as \defterm{$\pi_{[i,j]}$}.
        Note that if $i=j$, then $\pi_{[i,j]}=\pi_i$.
    \end{definition}
    
    \begin{definition}
        For an $n$-permutation $\pi$, we say that $\pi_{[i,j]}$ 
        is an \defterm{interval} if and only if it contains exactly the numbers in a contiguous interval.
        That is, if and only if 
        \[\{\pi_\ell\mid \ell\in [i,j]\} = [s,t]\] for some $s,t\in [n]$.
    \end{definition}
\end{frame}

\begin{frame}{Inflations}
    \begin{definition}
        Let $\sigma$ be a $k$-permutation and let $i_1, \, i_2,\, \dots,\, i_k$ be positive integers 
        whose sum is $n$.
        % where $i_1+i_2+\cdots i_k=n$.
        For $\ell\in [k]$, let $\alpha^{(\ell)}$ be an $i_\ell$-permutation.
        % of length $i_\ell$.
        We define the \defterm{inflation} of $\sigma$ by $\alpha^{(1)}, \,\alpha^{(2)},\,\dots,\, \alpha^{(k)}$
        (denoted $\defterm{\sigma[\alpha^{(1)},\, \alpha^{(2)},\,\dots, \,\alpha^{(k)}]}$)
        as the $n$-permutation $\pi$,
        % \[\pi=\defterm{\sigma[\alpha^{(1)}, \alpha^{(2)},\dots, \alpha^{(k)}]}\]
        where the following holds:\\\vspace{0.5cm}
        Given $s_0:=0$ and $s_\ell := i_1+i_2+\cdots i_\ell$ for $\ell\in [k]$,
        \begin{enumerate}[1.]
            \item the factors $\pi_{[s_{\ell-1}+1,\, s_{\ell}]}$ are intervals;
            \item $\text{red}(\pi_{[s_{\ell-1}+1,\, s_{\ell}]}) = \alpha^{(\ell)}$;
            \item $\pi_{s_t}<\pi_{s_u} \iff \sigma_t < \sigma_u$.
        \end{enumerate}
        \vspace{0.5cm}
        We call $\sigma$ a \defterm{quotient} of $\pi$.
        % and $\pi$ a \defterm{deflation} of $\sigma[\alpha^{(1)}, \,\alpha^{(2)},\,\dots,\, \alpha^{(k)}]$.
    \end{definition}
\end{frame}

\begin{frame}{Inflations - Example}
    % \begin{example}
        \begin{align*}
            \only<4>{
                \textcolor{orange}{\text{qu}}&\textcolor{orange}{\text{otient}}\\
                }
            \textcolor<2->{red}{8697}\textcolor<2->{blue}{41352}
            \only<3->{
                =&\textcolor<4>{orange}{21}[\textcolor{red}{3142},\textcolor{blue}{41352}]
                % =&\underbrace{
                %     \hbox{
                %         \textcolor{orange}{21}[\textcolor{red}{3142},\textcolor{blue}{41352}]
                %     }
                % }_{\hbox{inflation}
                % }
            }
            % \underbrace{\hbox{869741352}}_{\hbox{deflation}}
            % &\underbrace{\hbox{\textcolor{orange}{3142}[12, 12, 312, 1]}}_{\hbox{inflation}} = \underbrace{\hbox{45128673}}_{\hbox{deflation}}
        \end{align*}
        
        \begin{figure}
            \only<1>{\tikzfig{./}{869741352}}
            \only<2->{\tikzfig{./}{869741352p}}
            \caption{Graph of the permutation 869741352}
        \end{figure}
    % \end{example}
\end{frame}


% \begin{frame}{Inflations}
%     \begin{example}
%         \begin{align*}
%             \textcolor{orange}{\text{qu}}&\textcolor{orange}{\text{otient}}\\
%             &\underbrace{\hbox{\textcolor{orange}{21}[3142,41352]}}_{\hbox{inflation}} = 
%             \underbrace{\hbox{869741352}}_{\hbox{deflation}}
%             % &\underbrace{\hbox{\textcolor{orange}{3142}[12, 12, 312, 1]}}_{\hbox{inflation}} = \underbrace{\hbox{45128673}}_{\hbox{deflation}}
%         \end{align*}
%     \end{example}
%     \begin{figure}
%         \tikzfig{./}{869741352p}
%     \end{figure}
% \end{frame}

\begin{frame}{Inflations}
    Sometimes, there is more than one way to express a permutation as an inflation.\\\vspace{1cm}
    % \pause
    \begin{example}
        \begin{align*}
            34215
            &= 12[3421,1]\\
            &= 213[12,21,1]
        \end{align*}
    \end{example}
    \pause
    We seek a unique representation of permutations as inflations.
    % \begin{example}
    %     \begin{align*}
    %         657341289 
    %         % &= 12[6573412, 12]\\
    %         &= 213[43512,12, 12]\\
    %         &= 2314[21, 1, 21[12, 12], 12]
    %     \end{align*}
    %     \pause
    %     \begin{figure}
    %         \includegraphics[scale=0.5]{figures/tree.jpg}
    %         \caption{The substitution decomposition tree corresponding to the permutation 657341289.\cite{kitaev-textbook}}
    %     \end{figure}
    % \end{example}
\end{frame}


\begin{frame}{Simple Permutations}
    \begin{definition}
        An $n$-permutation is \defterm{simple} if and only if 
        its intervals are all of length 0, 1 or $n$.
    \end{definition}\vspace{1cm}
    \pause
    \begin{example}
        \begin{itemize}
            \item 2413
            \item 3142
            \item 524613
            % \item 68513742
            % \item 11 8 6 3 7 5 9 1 10 12 4 2
        \end{itemize}
    \end{example}
\end{frame}

\begin{frame}{Simple Permutations}
    \begin{prop}[Albert and Atkinson \cite{albert-atkinson}]
        Every permutation may be written as the inflation of a unique simple permutation. 
        Moreover, if $\pi$ can be written as $\sigma[\alpha^{(1)}, \alpha^{(2)}, \dots, \alpha^{(m)}]$ 
        where $\sigma$ is simple and $m\geq 4$, 
        then the $\alpha^{(i)}$s are unique.
    \end{prop}\vspace{1cm}
    \pause
    What about $m<4$?\\
    Note that all 3-permutations are not simple.
    So the only simple permutations of length less than 4 are 12 and 21.
\end{frame}

\begin{frame}{Sum/Skew Sum Decomposable Permutations}
    \begin{definition}
        \begin{itemize}
            \item Inflation of 12: \defterm{sum decomposable} ($\alpha\oplus \beta:=12[\alpha,\beta]$) 
            \item Inflation of 21: \defterm{skew sum decomposable} ($\alpha\ominus \beta:=21[\alpha,\beta]$) 
        \end{itemize}
    \end{definition}
    \begin{figure}
        \includegraphics[scale=0.5]{figures/SD-SSD.jpg}
        \caption{Illustration of $14325 \oplus 4231 = 143259786$ and $14325\ominus 4231 = 587694231$ using permutation matrices.\cite{kitaev-textbook}}
    \end{figure}
\end{frame}
    
\begin{frame}{Sum/Skew Sum Decomposable Permutations}
    \begin{definition}
        \begin{itemize}
            \item Not an inflation of 12: \defterm{sum indecomposable}
            \item Not an inflation of 21: \defterm{skew sum indecomposable}
        \end{itemize}
    \end{definition}
\end{frame}
    
\begin{frame}{Sum/Skew Sum Decomposable Permutations}
    \begin{prop}[Albert and Atkinson \cite{albert-atkinson}]
        If $\pi$ is an inflation of 12, 
        say $\pi=12[\alpha,\beta]$,  
        and $\alpha$ is sum indecomposable,
        then both $\alpha$ and $\beta$ are unique. 
        The same holds with 12 replaced by 21 and ``sum'' replaced by ``skew sum''.
    \end{prop}
\end{frame}
    
% \begin{frame}{Separable Permutations}
%     \begin{definition}
%         The \defterm{separable permutations} are those which can be obtained
%     by repeatedly applying the $\oplus$ and $\ominus$ operations on the permutation 1.
%     \end{definition}
%     \begin{figure}
%         \includegraphics[scale=0.45]{figures/separable.jpg}
%         \caption{Decomposition of 143259786 using $\oplus$ and $\ominus$ operations.\cite{kitaev-textbook}}
%     \end{figure}
% \end{frame}

% \begin{frame}{Separable Permutations}
%     \begin{theorem}[\textit{folklore}]
%         A permutation is separable if and only if it avoids 2413 and 3412.
%     \end{theorem}
%     \begin{corollary}
%         Simple permutations of length at least 4 contain 2413 or 3142, or both.
%     \end{corollary}
% \end{frame}

% \begin{frame}{Application of simple permutations to pattern/POP avoidance}
%     \begin{theorem}[Albert and Atkinson \cite{albert-atkinson}]
%         A permutation class with only finitely many simple permutations has a readily computable algebraic generating function. 
%         Moreover, if a class with finitely many simple permutations avoids a decreasing permutation of some length n, then the class is enumerated by a rational generating function.
%     \end{theorem}
% \end{frame}

\subsection{Research Motivation}

\begin{frame}{Applications of POP/pattern avoidance \cite[see][]{kitaev-textbook}}
    \begin{itemize}
        \item Computer Science: 
        \begin{itemize}
            \item Sorting permutations with stacks or other devices
            \item Planar maps and trees (graph theory)
        \end{itemize}
        \item Computational biology:
        \begin{itemize}
            \item Genome duplication-random loss model 
        \end{itemize}
        \item Statistical Mechanics
        \begin{itemize}
            \item Partially Asymmetric Simple Exclusion Process (PASEP)
        \end{itemize}
        \item Encoding combinatorial objects
        \begin{itemize}
            \item Schr\"oder permutations 
            \item Catalan numbers
            \item Dyck paths
            \item Schubert varieties and Kazhdan-Lusztig polynomials
        \end{itemize}
    \end{itemize}
\end{frame}

\begin{frame}{Motivation}
    ``The variety of different classical combinatorial objects related to a single pattern class hierarchy is rather striking. 
    Both the permutation patterns theory and the other structures involved benefit from the connection'' 
    - S. Kitaev \cite[see][p. 40]{kitaev-textbook}
\end{frame}


\begin{frame}{Literature}
    Gao and Kitaev \cite{gao-kitaev-2019} enumerated the avoidance sets for POPs of size 4 and 5,
    and observed that the sequence $(\abs{Av_n(P)})_{n\geq 1}$ already exists in The Online Encyclopedia of Integer Sequences (OEIS)
    for some POPs $P$.\\
    \vspace{0.3cm}
    They listed 15 POPs whose avoidance sets are enumerated by the same sequence as certain combinatorial objects.\\
    I studied 5 pairs and constructed explicit bijections.\\
    \vspace{0.3cm}
    
    \begin{center}
        \tiny \tikzfig{./}{Av13} \hspace{1cm} \tiny\tikzfig{./}{Av5} \\\vspace{0.5cm}
        \tiny\tikzfig{./}{tinyAv1} \hspace{1cm} \tikzfig{./}{P4} \hspace{1cm} \tikzfig{./}{R4} 
    \end{center}
\end{frame}

\begin{frame}{Main Question}
    \begin{table}
        \centering
        \begin{tabular}{|c|c|c|}
            \hline
                \# & \thead{Permutations\\avoiding the POP} & \thead{Equinumerous structures} \\ \hline
                1 & \makecell{\tikzfig{figures}{Av1}}  & \makecell{permutations \\avoiding the patterns \\ 2413, 2431, 4213, 3412, \\ 3421, 4231, 4321, 4312} \\
                2 & \makecell{\tikzfig{figures}{Av5}}  & \makecell{2-ary shrub forests\\of $n$ heaps avoiding\\the patterns 231, 312, 321}  \\
                3 & \makecell{\tikzfig{figures}{Av13}} & \makecell{number of ground-state \\ 3-ball juggling sequences \\ of period $n$} \\
            \hline
        \end{tabular}
        % \caption{List of objects for which interesting bijections were found \cite{gao-kitaev-2019}}
    \end{table}
\end{frame}

\begin{frame}{Main Question}
    \begin{table}
        \centering
        \begin{tabular}{|c|c|c|}
            \hline
            \# & \thead{Permutations\\avoiding the POP} & \thead{Equinumerous structures} \\ \hline
            4 & \makecell{\tikzfig{figures}{Av9}}  & \makecell{levels in all \\compositions of $n+1$ \\with only 1's and 2's} \\
            5 & \makecell{\tikzfig{figures}{Av10}} & \makecell{number of $n$-permutations \\ for which the partial sums \\of signed displacements \\ do not exceed 2}  \\
            \hline
        \end{tabular}
        \caption{List of objects with interesting bijections found \cite{gao-kitaev-2019}}
    \end{table}
\end{frame}

\section{Research}

% \subsection{Case Study: POP \texorpdfstring{$\lambda$}{lambda}}
\subsection{Bijections formed}
\subsubsection{A case study}
\begin{frame}{POP $\lambda$ - Sum indecomposables}
    \begin{theorem}
        Let 
        $P:=\{2413, 2431, 4213, 3412, 3421, 4231, 4321, 4312\}$
        and $\lambda: = \tiny\tikzfig{./}{tinyAv1} $.
        Let $I_k$ be the identity permutation $1\, 2\, \cdots \, k$ for all $k$.
        % Then $\abs{Av_n(\lambda)} = \abs{Av_n(P)}$ for all $n$
        Then the map
        \begin{align*}
            1&\mapsto 1\\
            \textcolor<2>{red}{21}\textcolor<3>{orange}{[I_{n-1},1]} &\mapsto \textcolor<2>{red}{21}\textcolor<3>{orange}{[1,I_{n-1}]}\\
            % 21[12\cdots (n-3)(n-1)(n-2),1] &\mapsto 21[1, 12\cdots (n-3)(n-1)(n-2)]\\
            \textcolor<2>{red}{231}\textcolor<3>{orange}{[I_{n-3},21,1]} &\mapsto \textcolor<2>{red}{312}\textcolor<3>{orange}{[1, I_{n-3},21]}\\
            \textcolor<2>{red}{21}\textcolor<3>{orange}{[I_{n-2},21]} &\mapsto \textcolor<2>{red}{231}\textcolor<3>{orange}{[21,I_{n-3},1]}\\
            % 21[12\cdots \ell (\ell+2)\cdots (n-2)(\ell+1)]&=\\
            % =21[132[I_{\ell},I_{n-\ell-2},1],1] &=\\
            \textcolor<2>{red}{2431}\textcolor<3>{orange}{[I_{\ell},I_{n-\ell-2},1,1]} &\mapsto \textcolor<2>{red}{41352}\textcolor<3>{orange}{[1,I_{\ell}, 1, I_{n-\ell-3},1]} \text{for all }\ell \in [n-4]\\
            \textcolor<2>{red}{2413}\textcolor<3>{orange}{[I_{\ell},I_{n-\ell-2},1,1]} &\mapsto \textcolor<2>{red}{3142}\textcolor<3>{orange}{[1,I_{\ell}, I_{n-\ell-2},1]} \text{for all } \ell\in [n-3]
        \end{align*}
        is a bijection from the \textbf{sum indecomposable} permutations in $Av_n(\lambda)$ to the \textbf{sum indecomposable} permutations in $Av_n{(P)}$.
    \end{theorem}
    \only<4>{}
\end{frame}

\begin{frame}{POP $\lambda$ - Sum decomposables}
    Let $g_n$ be the map on sum indecomposable $n$-permutations as defined in the previous slide for all $n$.
    Define the recursive function 
    \begin{align*}
        f_n(\pi) = 
        \begin{cases}
            g_n(\pi)&\text{if }\pi\text{ is sum indecomposable}\\
            12[f_n(\alpha), f_n(\beta)] &\text{if }\pi=12[\alpha,\beta]
        \end{cases}
    \end{align*}
    Then for $n\geq 1$, $f_n$ is a bijection from $Av_n(\lambda)$ to $Av_n{(P)}$.
\end{frame}

\subsubsection{Others}

\begin{frame}{Juggling Sequences
    \tikzoverlay[text width=3cm] at (2.5cm,1cm) {
    \begin{figure}
        \includegraphics[scale=0.12]{figures/juggling}
    \end{figure} 
    };
    \footnote{\tiny{Juggle Clipart Black And White \#23791505 [JPG]. Retrieved from https://www.clipart.email/download/23791505.html.}}
    }
    \begin{theorem}
    Given positive integers $n$ and $b$,
    let $\theta$ be the function defined on the set of ground state juggling sequences of period $n$ using $b$ balls
    to the set $Av_n\left(\tiny\tikzfig{./}{tinyQb2}\right)$,\\
    given by $\theta((t_1,t_2,\dots,t_n))=\pi$ 
    where $\pi_{t_i+i-b}=i$ for all $i\in [n]$.\\
    Then $\theta$ is a bijection.
    \end{theorem}
\end{frame}

\begin{frame}{2-ary Shrub Forests
    \tikzoverlay[text width=0.1cm] at (2cm,0.5cm) {\tiny\tikzfig{./}{shrub}};}
    \begin{theorem}
    Define $\mathcal{P}_{3n}$ as the set of \defterm{2-ary shrub forests of $n$ heaps} avoiding the patterns 231, 312 and 321.
    The function $\theta: Av_n(Q_4)\rightarrow \mathcal{P}_{3n-3}$ given by 
    \begin{align*}
        \theta(12)&=123, \quad \theta(21)=132, \quad \text{and for }\abs{\pi}=n\geq 3,\\
        % \theta\left(\pi_{[1,n-1]}\,n\right ) &= 1 \oplus 132 \oplus \theta\left(\pi_{[1,n-1]}\right)_{[2,3n]},\\
        % \theta\left(\pi_{[1,n-2]}\,n\,\pi_n\right ) &= 1 \oplus 123 \oplus \theta\left(\pi_{[1,n-2]}\,n\,\pi_n\right )_{[2,3n]},\\
        % \theta\left(\pi_{[1,n-3]}\,n\,\pi_{[n-1,n]}\right ) &= 1 \oplus 213 \oplus \theta\left(\pi_{[1,n-3}\,n\,\pi_{[n-1,n]} \right )_{[2,3n]}.
        \theta(\pi) &= 
        \begin{cases}
            1 \oplus 132 \oplus \theta\left(\pi_{[1,n-1]}\right)_{[2,3n]}, &\text{if }\pi_n=n,\\
            1 \oplus 123 \oplus \theta\left(\pi_{[1,n-2]}\,\pi_n\right )_{[2,3n]},&\text{if }\pi_{n-1}=n,\\
            1 \oplus 213 \oplus \theta\left(\pi_{[1,n-3]}\,\pi_{[n-1,n]} \right )_{[2,3n]} &\text{if }\pi_{n-2}=n.
        \end{cases}
    \end{align*}
    is a bijection.
    \end{theorem}
\end{frame}

\begin{frame}{Compositions}
    Denote \defterm{$\mathcal{C}_n$} as the \defterm{set of compositions of $n$ of ones and twos}.\\\vspace{1cm}
    \begin{theorem}
        Let $f:\mathcal{C}_n\rightarrow Av_n\left(\tiny\tikzfig{./}{P3}\right)$ 
    defined as 
    \[f(r_1+r_2+\cdots+r_k)=123\cdots k[\alpha_1,\alpha_2,\dots,\alpha_k]\]
    where $r_1+r_2+\cdots+r_k$ is a composition of $n$ of ones and twos,
    and 
    \[\alpha_i=\begin{cases}
        1&\text{if }r_i=1\\
        21&\text{if }r_i=2
    \end{cases}.\]
    Then $f$ is a bijection.
    \end{theorem}
\end{frame}
    
\begin{frame}{Marked Compositions}
    Denote the \defterm{set of marked compositions of $n$} by \defterm{$\mathcal{L}_n$}.
    \begin{theorem}
        Let $g:\mathcal{L}_{n+1} \rightarrow Av_n\left(\tiny\tikzfig{./}{P4}\right)$, where $n\geq 0$ and $r_i\in \{1,2,\overline{1},\overline{2}\}$ for $i\in [k]$ and $k\geq 1$
        % \begin{align*}
        %     1 &\mapsto 1\\
        %     \overline{1+1} &\mapsto 21\\
        %     r_1+r_2+\cdots+r_{k-1}+1 &\mapsto 12[1,g(r_1+\cdots +r_{k-1})]
        % \end{align*}
    \begin{align*}
        &g(r_1+r_2+\cdots r_k)=\\
        &=\begin{cases}
            1&\text{ if } k=1,\,r_1=1,\\
            21&\text{ if } k=2,\quad r_{1}=r_2=\overline{1},\\
            12[1,g(r_1+\cdots +r_{k-1})]&\text{ if } r_k=1,\\
            12[21,g(r_1+\cdots +r_{k-1})]&\text{ if } r_k=2,\\
            21[f(r_1+\cdots +r_{k-2}),1]&\text{ if } r_{k-1}=r_k=\overline{1},\\
            3142[1,1,f(r_1+\cdots +r_{k-2}),1]&\text{ if } r_{k-1}=r_k=\overline{2}.
        \end{cases}
    \end{align*}
    Then $g$ is a bijection.
    \end{theorem}
\end{frame}

\begin{frame}{Partial Sums of Signed Displacements}
    \begin{theorem}
        Let $\mathfrak{S}_n$ be the set of $n$-permutations for which the \textit{partial sums of signed displacements do not exceed 2}.\\
        Then the map $\theta: Av_n\left(\tiny\tikzfig{./}{R4}\right)\rightarrow \mathfrak{S}_n$ given by 
        \[\theta(\pi) = 
        \begin{cases}
            \pi                 &\text{ if }\abs{\pi}\in[4]\\
            \beta \oplus 3142 &\text{ if }\pi=2413\oplus \beta \text{ for some permutation }\beta\\
            \beta \oplus \alpha &\text{ if } \pi=\alpha\oplus\beta \text{ for some permutations }\alpha\text{ and }\beta, \\
            &\text{ where }\alpha\neq 2413.
        \end{cases}\] 
        is a bijection.
    \end{theorem}
\end{frame}

\subsection{Enumeration of simples avoiding 2413, 3412, 3421}

\begin{frame}{Fibonacci Simples}
    Unlike the previous POPs studied, this set of patterns are avoided by infinitely-many simple permutaitons.

    \begin{theorem} 
        Let $P$ be a pattern, a set of patterns, or a POP.
        Denote $Av_n^S(P)$ as the set of simple $n$-permutations avoiding $P$.\\
        \vspace{0.7cm}
        For $n\geq 3$, 
        $\abs{Av_n^S(2413,3412,3421)}=F(n-3)$,
        where $F(0)=0$, $F(1)=1$ and $F(k)=F(k-1)+F(k-2)$ for $k\geq 2$.
        % Moreover, $Av_n^S(2413,3412,3421)$ can be partitioned into the following sets:
        % \begin{table}
        %     \begin{tabular}{|c|cc|}
        %         \hline
        %         Set & \multicolumn{2}{c|}{Form of permutation} \\\hline
        %         $A_n$ & $(n-1)\,1\,(n-2)\,\cdots\, n\,k$ & where $k\in [3,n-3]$ \\    
        %         $B_n$ & $(n-1)\,(n-3)\,\cdots \,(n-2)\,n\,k$ & where $k\in [2,n-3]$ \\    
        %         $C_n$ & $(n-1)\,1\,(n-3)\,\cdots \,(n-2)\,n\,k$ & where $k\in [3,n-3]$ \\\hline
        %     \end{tabular}
        %     \label{table:fib-summary}
        %     \caption{Summary of the types of permutations in $Av_n^S(2413,3412,3421)$ for $n\geq 5$}
        % \end{table}
    \end{theorem} 
\end{frame}

\subsection{Future work}

\begin{frame}{Future work - Fibonacci Simples}
    Observation: $Av_n^S(2413,3412,3421)=Av_n^S(2413,3412,3421,\textcolor<2>{red}{2431})$ for $n\leq 25$.
    I conjecture that the same holds for all $n$.\vspace{1cm}
    
    This observation leads us to an interesting question: \\
    Which avoidance sets have the same set of simples?
\end{frame}

\begin{frame}{Future work - levels of compositions of ones and twos}
Define $P_k$ to be the POP \tikzfig{./}{Pk}.
I studied $P_3$ and $P_4$.
\begin{enumerate}[1.]
    \item Are there any combinatorial objects that have a natural bijective relationship with the avoidance set of $P_k$ for any $k\geq 5$?
    \item Is there a POP whose avoidance set is in bijection with the levels in compositions of ones, twos and threes?
    \item Are there any combinatorial objects that have a natural bijective relationship with the avoidance set 
    of the POP with $k$ elements labelled $1,2,\dots,k$ where $1>3>5$,
    or, more generally, with  $1>3>\cdots > 2i+1$ for some $i\geq 2$? 
\end{enumerate}
\end{frame}

\begin{frame}{Future work - partial sums of signed displacements}
Define $R_k$ to be the POP \tikzfig{./}{Rk}.\\
I studied $R_4$.
\begin{enumerate}
    \item What is the size of $Av_n(R_k)$ for all $k\geq 6$, $n\geq 1$?
    \item What is the size of $\mathfrak{S}_{k,n}$, which we define as the set of permutations whose partial sums of signed displacements do not exceed $k$, for all $k\geq 3$?
    \item Are there other $k$ and $\ell$ such that $\abs{\mathfrak{S}_{k,n}}=\abs{Av_n(R_\ell)}$ for all $n\geq 1$?
    % \item Gao and Kitaev \cite{gao-kitaev-2019} observed that sequence enumerating $Av_{n-1}(R_4)$
    % corresponds to sequence \href{https://oeis.org/A232164}{A232164} as well for $n\geq 1$.
    % The latter sequence counts the number of 
    % Weyl group elements, not containing an $s_r$ factor, which contribute nonzero terms to Kostant's weight multiplicity formula when computing the multiplicity of the zero-weight in the adjoint representation for the Lie algebra of type $C$ and rank $n$.
    % Using our analysis on the set $Av(R_4)$, can we construct a natural bijection between the two relevant sets?
\end{enumerate}
\end{frame}

\section{Bibliography}
\begin{frame}{Bibliography}
    \printbibliography
\end{frame}


\appendix
\section<presentation>*{\appendixname}

\begin{frame}{POP $\lambda$}
    Let $P=\{2413, 2431, 4213, 3412, 3421, 4231, 4321, 4312\}$
    and $\lambda = \tiny\tikzfig{./}{tinyAv1} $.
    Observation: $\abs{Av_n(\lambda)} = \abs{Av_n(P)}$ for all $n$. \\\vspace{1cm}
    Step 1:
    \begin{table}
        \begin{tabular}{c|c}
            \thead{Simple permutations\\avoiding $\lambda$} & \thead{Simple permutations\\avoiding $P$}\\ \hline 
            12, 21, 2413 & 12, 21, 3142, 41352
        \end{tabular}
    \end{table}
\end{frame}
    
\begin{frame}{POP $\lambda$: Step 2}
    \begin{table}
        \begin{tabular}{c|c}
            \thead{Size} & \thead{Subset of $Av_n(\lambda)$} \\\hline
            $n$   & Inflations of 21 \\
            $n-3$ & Inflations of 2413 ($n\geq 4$)\\
            $\sum_{i=1}^{n-1}(2i-3) \abs{Av_{n-i}(\lambda)}$ & Inflations of 12
        \end{tabular}\vspace{1cm}
        \begin{tabular}{c|c}
            \thead{Size} & \thead{Subset of $Av_n(P)$} \\\hline
            4     & Inflations of 21 \\
            $n-4$ & Inflations of 41352 ($n\geq 4$)\\
            $n-3$ & Inflations of 3142\\
            $\sum_{i=1}^{n-1}(2i-3) \abs{Av_{n-i}(P)}$ & Inflations of 12
        \end{tabular}
    \end{table}
    \begin{center}
        $\abs{Av_n(\lambda)} = \sum_{i=1}^{n} \abs{Av_{n-i}(P)} = \abs{Av_n(P)}$ for all $n$. 
    \end{center}
\end{frame}

\begin{frame}{POP $\lambda$: Step 3}
    \begin{table}
        \begin{tabular}{c|c}
            \thead{Simple quotient} & \thead{Sum indecomposable\\permutations in $Av_n(\lambda)$} \\\hline
            21   & \makecell{$21[I_{n-1},1],$\\$ 21[I_{n-2},21],$\\$ 231[I_{n-3},21,1],$\\$ 2431[I_{\ell},I_{n-\ell-2},1,1], \ell=2,3,\dots,n-1$} \\
            2413 & $ 2413[I_{\ell},I_{n-\ell-2},1,1] $, $\ell\in [n-3]$  
        \end{tabular}\vspace{0.5cm}
        \begin{tabular}{c|c}
            \thead{Simple quotient} & \thead{Sum indecomposable\\permutations in $Av_n(P)$} \\\hline
            21     & \makecell{$21[1,I_{n-1}],$\\$312[1,I_{n-3},21]$}\\
            41352  & $41352[1,I_\ell, 1,I_{n-\ell-3},1]$, $\ell\in [n-4]$\\
            3142   & $3142[1,I_{\ell},I_{n-\ell-2},1]$, $\ell\in [n-3]$\\
        \end{tabular}
    \end{table}
\end{frame}

\begin{frame}{Levels in Compositions}
    \begin{definition}
        Denote \defterm{$\mathcal{C}_n$} as the \defterm{set of compositions of $n$ of ones and twos}.\\\vspace{1cm}
        % 
        A \defterm{level} in a composition of $n$ of ones and twos is a pair of consecutive ones or twos separated by a $+$ sign.
        I define a \defterm{marked composition} of $n$ to be a composition of $n$ 
        with exactly one level marked by adding a line above the pair of ones or twos. 
        I may denote a single summand that is part of a level by including a line above it,
        i.e. $\overline{1}+\overline{1}=\overline{1+1}$ and $\overline{2}+\overline{2}=\overline{2+2}$.\\\vspace{1cm}
        % 
        Denote the \defterm{set of marked compositions of $n$} by \defterm{$\mathcal{L}_n$}.
    \end{definition}
\end{frame}


\begin{frame}{Juggling Sequences}
    \begin{table}
        \centering
        \begin{tabular}{|c|c|c|}
            \hline
            \thead{State\\period, \# balls\\$J(n,b)$} & \thead{Juggling\\Sequence\\$T=(t_1,\dots,t_n)$} & \thead{$\pi\in Av_n(Q_{b+2})$\\$\pi_{t_i+i-b} = i$} \\\hline
            \makecell{$\sigma=(1,0)$\\$n=2$, $b=1$\\$J(n,b)=2$}      & \makecell{(1,1)\\(2,0)}  & \makecell{$12\inv=12$\\$21\inv=21$} \\
            \makecell{$\sigma=(1,0,0)$\\$n=3$, $b=1$\\$J(n,b)=4$}    & \makecell{(1,1,1)\\(3,0,0)\\(2,0,1)\\(1,2,0)} & \makecell{$123\inv=123$\\$312\inv=231$\\$213\inv=213$\\$132\inv=132$} \\
            % \makecell{$\sigma=(1,0,0,0)$\\$n=4$, $b=1$\\$J(n,b)=8$}  & \makecell{(1,1,1,1)\\(3,0,0,1)\\(2,0,1,1)\\(1,2,0,1)\\(2,0,2,0)\\(4,0,0,0)\\(1,3,0,0)\\(1,1,2,0)} & \makecell{$1234\inv=1234$\\$3124\inv=2314$\\$2134\inv=2134$\\$1324\inv=1324$\\$2143\inv=2143$\\$4123\inv=2341$\\$1423\inv=1342$\\$1243\inv=1243$}\\
            % \makecell{$\sigma=(1,1)$\\$n=2$, $b=2$\\$J(n,b)=2$}      & \makecell{(3,4)\\(4,3)} & \makecell{$12\inv=12$\\$21\inv=21$} \\
            % \makecell{$\sigma=(1,1,0)$\\$n=3$, $b=2$\\$J(n,b)=6$}    & \makecell{$123\inv=123$\\$132\inv=132$\\$213\inv=213$\\$231\inv=312$\\$312\inv=231$\\$321\inv=321$} \\
            \hline
        \end{tabular}
        \label{table:avQ-juggling}
        \caption{Sample values for juggling sequences and their image under $\theta$ for small $n$ and $b$.}
    \end{table}
\end{frame}

\begin{frame}{Juggling Sequences}
    \begin{figure}
        \includegraphics[scale=0.5]{figures/juggling-state.jpg}
        \caption{The juggling sequence $(4,1,5,2)$ with its associated state $\sigma=\langle 1,1,0,1\rangle$. \cite{chung-graham}}
    \end{figure}
\end{frame}
\end{document}