% !TEX root = ../main.tex
\documentclass[../main.tex]{subfiles}
%
\thispagestyle{plain}
% \begin{center}
% \Large
% \textbf{Thesis Title}
    
% \vspace{0.4cm}
% \large
% Thesis Subtitle
    
% \vspace{0.4cm}
% \textbf{Kai Ting Keshia Yap}
    
% \vspace{0.9cm}
% \textbf{Abstract}
% \end{center}

A permutation $\pi$ contains a pattern $\sigma$
if and only if there is a subsequence in $\pi$
with its letters are in the same relative order as those in $\sigma$.
Partially ordered patterns (POPs) provide a convenient way to denote patterns 
in which the relative order of some of the letters does not matter. 
This thesis elucidates connections between the avoidance sets of a few POPs 
with other combinatorial objects,
directly answering five open questions posed by Gao and Kitaev \cite{gao-kitaev-2019}.
This was done by thoroughly analysing the avoidance sets and developing 
recursive algorithms to derive these sets and their corresponding combinatorial objects in parallel,
which yielded a natural bijection.
We also analysed an avoidance set whose simple permutations are enumerated by the Fibonacci numbers
and derived an algorithm to obtain them recursively.\\

\noindent Keywords: bijection, pattern avoidance, permutation, POP avoidance, simple permutation
